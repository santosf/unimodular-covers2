
\documentclass{amsart}

\textheight=22.5cm
\topmargin=0.5cm

\usepackage{amssymb}
\usepackage{graphicx,color}
\usepackage{amsmath}
\usepackage{url}
\usepackage[colorlinks=true]{hyperref}
\usepackage{doi}
%\usepackage{pictex}



\usepackage{cleveref}
\Crefname{counterexample}{counterexample}{counterexamples}
\Crefname{counterexample}{Counterexample}{Counterexamples}
\Crefname{conjecture}{conjecture}{conjectures}
\Crefname{conjecture}{Conjecture}{Conjectures}


\usepackage{tikz-cd}
\usepackage{tikz}
\usetikzlibrary{matrix}
\usepackage{tkz-euclide}
\usetikzlibrary{intersections}
%\usetkzobj{all}


\newcommand{\limites}{}

\theoremstyle{plain}
\newtheorem{theorem}{Theorem}[section]
\newtheorem{proposition}[theorem]{Proposition}
\newtheorem{lemma}[theorem]{Lemma}
\newtheorem{corollary}[theorem]{Corollary}
\newtheorem{claim}[theorem]{Claim}

\theoremstyle{definition}
\newtheorem{definition}[theorem]{Definition}
\newtheorem{notation}[theorem]{Notation}
\newtheorem{example}[theorem]{Example}
\newtheorem{problem}[theorem]{Problem}
\newtheorem{question}[theorem]{Question}
\newtheorem{conjecture}[theorem]{Conjecture}
\newtheorem{remark}[theorem]{Remark}


% Macros in general

\newcommand{\F}{ \ensuremath{\mathbb{F}}}
\newcommand{\N}{ \ensuremath{\mathbb{N}}}
\newcommand{\Z}{ \ensuremath{\mathbb{Z}}}
\newcommand{\C}{ \ensuremath{\mathbb{C}}}
\newcommand{\R}{ \ensuremath{\mathbb{R}}}

\newcommand{\T}{ \ensuremath{\mathcal{T}}}

\newcommand{\bb}{{\mathbf{b}}}
\newcommand{\cc}{{\mathbf{c}}}
\newcommand{\ttt}{\mathbf{t}}
\newcommand{\GL}{{GL}_r (K)}

\renewcommand{\int}{\operatorname{int}}
\newcommand{\width}{\operatorname{width}}
\newcommand{\ind}{\operatorname{index}}
\newcommand{\Vol}{\operatorname{Vol}}
\newcommand{\area}{\operatorname{area}}
\newcommand{\length}{\operatorname{length}}

\renewcommand{\vec}[1]{\overrightarrow#1}
\newcommand{\vecline}[1]{\langle \vec #1 \rangle}


\newcommand{\cone}{\ensuremath{\mathrm{cone}}\hspace{1pt}}
\newcommand{\conv}{\ensuremath{\mathrm{conv}}\hspace{1pt}}
\newcommand{\cayley}{\operatorname{Cay}}
\newcommand{\rank}{\operatorname{rank}}
\newcommand{\reg}{\operatorname{reg}}

\newcommand{\Image}{\ensuremath{\mathrm{Im}}\hspace{1pt}}

\usepackage[colorinlistoftodos]{todonotes}

\newcommand{\giulia}[1]{\todo[size=\tiny,color=blue!30]{#1 \\ \hfill --- G.}}
\newcommand{\Giulia}[1]{\todo[size=\tiny,inline,color=blue!30]{#1 \\ \hfill --- G.}}

\newcommand{\paco}[1]{\todo[size=\tiny,color=green!30]{#1 \\ \hfill --- P.}}
\newcommand{\Paco}[1]{\todo[size=\tiny,inline,color=green!30]{#1 \\ \hfill --- P.}}



\date{\today}

%\author[G.~Codenotti]{Giulia Codenotti}
%\author[F.~Santos]{Francisco Santos}

%\address[G.~Codenotti]
%{
%Institut f\"ur Mathematik, Freie Universit\"at Berlin, Germany
%}
%\email{codenotti@math.uni-frankfurt.de}

%\address[F.~Santos]
%{
%Department of Mathematics, Statistics and Computer Science, University of Cantabria, Spain
%}
%\email{francisco.santos@unican.es}

%\thanks{The authors were supported by the Einstein Foundation Berlin under grant EVF-2015-230. 
%Work of F. Santos is also supported by grants PID2019-106188GB-I00 funded by MCIN/AEI/10.13039/501100011033 and by FPU19/04163 of the Spanish Government and by project CLaPPo (21.SI03.64658) of Universidad de Cantabria and Banco Santander}


%%%%%%%%%%%%%%%%%%%%%%%%%%%%%%%%%%%%%%%%%%%%%%%%%%%%%%%%%%%

\title{Unimodular covers of 3-dimensional parallelepipeds and Cayley sums}
\author{}

\begin{document}

\begin{abstract}
We show that the following classes of lattice polytopes have unimodular covers, in dimension three:  parallelepipeds,  smooth centrally symmetric polytopes, and  Cayley sums $\cayley(P,Q)$ where the normal fan of $Q$ refines that of $P$. This improves results of Beck et al.~(2018) and Haase et al.~(2008) where the last two classes were shown to be IDP.
\end{abstract}

\maketitle


\section{Introduction}

A lattice polytope $P\subset \R^d$ has the \emph{integer decomposition property} if for every positive integer $n$, every lattice point $p \in nP\cap \Z^d$ can be written as a sum of $n$ lattice points in $P$. We abbreviate this by saying that ``$P$ is IDP''. Being IDP is interesting in the context of  both enumerative combinatorics (Ehrhart theory) and algebraic geometry (projective normality of toric
varieties). It falls into a hierarchy of several properties each stronger than the previous one; see, e.g., \cite[Section 2.D]{BGbook}, \cite[Sect. 1.2.5]{HPPS-survey}, \cite[p. 2097]{mfo2004}, \cite[p. 2313]{mfo2007}.
Let us here only mention that
\[
P \text{ has a unimodular triangulation}\Rightarrow
P \text{ has a unimodular cover}\Rightarrow
P \text{ is IDP.}
\]
Remember that a \emph{unimodular triangulation} is a triangulation of $P$ into unimodular simplices, and a \emph{unimodular cover} is a collection of unimodular simplices whose union equals $P$. 

Oda \cite{Oda1997} posed several questions regarding smoothness and the IDP property for lattice polytopes.
%
Following \cite{HaaseHof, Tsuchiya}, we say that a pair $(P, Q)$ of lattice polytopes has the integer decomposition property, or that \emph{the pair $(P,Q)$ is IDP}, if 
\begin{align*}
\label{eq:mixedIDP}
(P+Q) \cap \Z^d = P \cap \Z^d + Q \cap \Z^d,
\end{align*}
where $A+B :=\{a+b: a\in A, b\in B\}$ denotes the \emph{Minkowski sum} of two sets $A,B\subset \R^d$.
\paco{added def of Minkowski sum}

A lattice polytope $Q$ is called \emph{smooth} if it is simple and the primitive edge directions at every vertex form a linear basis for the lattice; equivalently, if the projective toric variety defined by the normal fan of $Q$ is smooth. 
The following versions of Oda's questions are now considered conjectures~\cite{HNPS2008,mfo2007}, and they are open even in dimension three:
\begin{conjecture}
\label{conj:Oda}
\begin{enumerate}
\item 
\label{itm:smoothIDP}
(Related to problems 2 and 5 in \cite{Oda1997})
Every smooth lattice polytope is IDP.
\item 
\label{itm:mixedIDP}
(Related to problems 1, 3, 4, 6 in \cite{Oda1997}) Every pair $(P,Q)$ of lattice polytopes with $Q$ smooth and the normal fan of $Q$ refining that of $P$ is IDP.
\end{enumerate}
\end{conjecture}

When the normal fan of a polytope $Q$ refines that of another polytope $P$, as in the second conjecture, we say that $P$ \emph{is a weak Minkowski summand of $Q$}, since this is easily seen to be equivalent to the existence of a polytope $P'$ such that $P+P' = k Q$ for some dilation constant $k>0$. 
This property has the following algebraic implication for the projective toric variety $X_Q$: $P$ is a weak Minkowski summand of $Q$ if and only if the Cartier divisor defined by $P$ on  $X_Q$ is \emph{numerically effective}, or ``nef'' (see~\cite[Cor.~6.2.15, Thm.~6.3.12]{CLS}, but observe that what we here call  ``weak Minkowski summand'' is called ``$\N$-Minkowski summand'' there). 


\medskip 
Motivated by these and other questions, several authors have studied the IDP property for different classes of lattice polytopes,
with special attention to dimension $3$ (in dimension 2 it is straightforward that every lattice polygon has unimodular triangulations). 
%
For example,  very recently
Beck et al.~\cite{BHHHJKM2019} proved that all smooth centrally symmetric $3$-polytopes are IDP.
More precisely, they show that any such polytope can be covered by lattice 
parallelepipeds (affine images of $3$-cubes) and unimodular simplices, both of which are trivially IDP.
%
In \Cref{sec:parallelepipeds} we show:

\begin{theorem}
\label{thm:parallelepipeds}
Every $3$-dimensional lattice parallelepiped has a unimodular cover.
\end{theorem}

This, together with the mentioned result from~\cite{BHHHJKM2019}, gives:

\begin{corollary}
\label{coro:3cs}
Every smooth centrally symmetric lattice $3$-polytope has a unimodular cover. 
\qed
\end{corollary}

These results leave open the following important questions regarding parallelotopes:

\begin{question}
Do $3$-dimensional parallelepipeds have unimodular triangulations?
\end{question}

\begin{question}
Higher dimensional parallelotopes (affine images of cubes) are IDP. Do they have unimodular covers? 
\end{question}


The two-dimensional case of \Cref{conj:Oda}\eqref{itm:mixedIDP} is known to hold, with three different proofs by Fakhruddin~\cite{Fakhruddin}, Ogata~\cite{Ogata} and Haase et al.~\cite{HNPS2008}. This last one actually shows that smoothness of $Q$ is not needed. In dimension three, however, the conjecture fails without the smoothness assumption. Indeed, if we let $P=Q$ be any non-unimodular \emph{empty tetrahedron}, then $P$ is obviously a weak Minkowski summand of $Q$ but the pair $(P, Q)$ is not IDP. By an empty tetrahedron we mean a lattice tetrahedron containing no lattice points other than its vertices (see the proof of \Cref{lemma:corner} for a classification of them).

An alternative approach to \Cref{conj:Oda}\eqref{itm:mixedIDP} is via Cayley sums, which we discuss in  \Cref{sec:cayley}. 
Recall that the \emph{Cayley sum} of two lattice polytopes $P,Q\subset \R^d$ is the lattice polytope
\[
\cayley(P,Q) := \conv(P\times\{0\} \cup Q \times \{1\}) \subset \R^{d+1}.
\]
We normally require $\cayley(P,Q)$ to be full-dimensional (otherwise we can delete coordinates) but $P$ or $Q$ do not necessarily need to be fulldimensional. We only require the linear subspaces parallel to them to span $\R^d$.

As we note in \Cref{prop:mixedIDP}, if the Cayley sum of $P$ and $Q$ is IDP then the pair $(P,Q)$ is IDP.
Hence, the following statement proved in \Cref{sec:cayley} is stronger than the afore-mentioned result of \cite{Fakhruddin,HNPS2008,Ogata}:

\begin{theorem}
\label{thm:cayley}
Let $Q$ be lattice polygon, and $P$ a weak Minkowski summand of $Q$. Then the Cayley sum $\cayley(P,Q)$ has a unimodular cover.
\end{theorem}

This has the following consequence, also proved in \Cref{sec:cayley}.
Here a  \emph{prismatoid} is a polytope whose vertices all lie in two parallel facets. 


\begin{corollary}
\label{coro:prismatoid}
Every smooth $3$-dimensional lattice prismatoid has a unimodular cover.
\end{corollary}

Let us mention that recent work of Gubeladze~\cite{Gubeladze} shows another class of 3-polytopes admitting unimodular covers: the convex hulls of all lattice points inside an ellipsoid; these had previously been shown to be IDP by Bruns, Gubeladze and Michalek~\cite{BGM}.
To date there are no known examples of IDP $3$-polytopes without a unimodular cover, although such polytopes exist in higher dimension~\cite{BG}.


%\subsection*{Acknowledgements:} We would like to thank Akiyoshi Tsuchiya, Spencer Backman, and Johannes Hofscheier for posing these questions to us and Christian Haase for helpful discussions.


\section{Parallelepipeds}
\label{sec:parallelepipeds}

The main tool for the proof of \Cref{thm:parallelepipeds} is what we call the parallelepiped circumscribed to a given tetrahedron, defined as follows:

\begin{definition}
\label{def:circunpara}
Let $T$ be a tetrahedron with vertices $p_1$, $p_2$, $p_3$, and $p_4$. Consider the points $q_i= \frac12 (p_1+p_2+p_3+p_4) - p_i$ for each $i\in \{1,2,3,4\}$, and let
\[
C(T)=\conv(p_i,q_i: i\in\{1,2,3,4\}).
\] 
$C(T)$ is a parallelepiped with facets $\conv(p_i, p_j, q_k, q_l)$ for all choices of $\{i,j,k,l\}=\{1,2,3,4\}$. We call it the \emph{parallelepiped circumscribed} to $T$.

For each $i \in \{1,2,3,4\}$, let $T_i=\conv(q_i, p_j, p_k, p_l)$, with $\{i,j,k,l\}=\{1,2,3,4\}$; we call the $T_i$ \emph{corner tetrahedra} of $C(T)$. Together with $T$ they triangulate $C(T)$.
\end{definition}

Modulo an affine transformation, the situation of $T$ and $C(T)$ is exactly that of the regular tetrahedron inscribed in a cube; see \Cref{fig:circumscribed_parall}. 
%
\begin{figure}[htb]
\includegraphics[scale=.25]{circumscribed_parall}
\caption{In red we have a tetrahedron $T$, in black its circumscribed parallelepiped $C(T)$, and in blue the corner simplex $T_4$.}
\label{fig:circumscribed_parall}
\end{figure}

Observe that the points $q_i$ need not be lattice points. However, the following lemma shows that we can find lattice points in each corner tetrahedron.
\begin{lemma}
\label{lemma:corner}
Let $T=\conv\{p_1,p_2,p_3,p_4\}$ be an empty lattice tetrahedron that is not unimodular. Let $C(T)$ be the parallelepiped circumscribed to $T$ and let $T_1, T_2,T_3$ and $T_4$ be the corresponding corner tetrahedra in $C(T)$. Then, every $T_i$ contains at least one lattice point different from $\{p_1,\dots,p_4\}$.
\end{lemma}

\begin{proof}
By White's classification of empty tetrahedra (\cite{White1964}, see also, e.~g.~\cite[Sect.~4.1]{HPPS-survey}), there is no loss of generality in assuming $T=\conv(p_1,p_2,p_3,p_4)$ with
\[
p_1=(0,0,0), \quad
p_2=(1,0,0), \quad
p_3=(0,0,1), \quad
p_4=(a,b,1).
\]
where $b\ge 2$ is the (normalized) volume of $T$, and $a\in \{1,\dots,b-1\}$ satisfies $\gcd(a,b)=1$. This gives 
\begin{align*}
q_1=\left(\frac{1+a}2,\frac{b}2,1\right), &&
q_2=\left(\frac{a-1}2,\frac{b}2,1\right), \quad\\
q_3=\left(\frac{1+a}2,\frac{b}2,0\right), &&
q_4=\left(\frac{1-a}2,-\frac{b}2,0\right).
\end{align*}

Then, the inequalities $b\ge 1+a \ge 2$ imply:
\[
u:=(1,1,0)\in \conv(p_1,p_2,q_3) \subset T_3, \quad
v:=(0,-1,0)\in \conv(p_1,p_2,q_4) \subset T_4.
\]

Observe that $u+v=p_1+p_2=q_3+q_4$.
Now, this implies that the quadrilateral $\conv(p_1,q_4,p_2,q_3)$ contains a fundamental domain for the lattice $\Z^2\times\{0\}$. Hence, its translate $\conv(q_2,p_3,q_1,p_4)$ contains a fundamental domain for $\Z^2\times\{1\}$ and, in particular, it contains at least one lattice point other than $p_3$ and $p_4$. By central symmetry around its center $\left(\frac{a}2,-\frac{b}2,1\right)$, $\conv(q_2,p_3,q_1,p_4)$ must contain lattice points in both triangles $\conv(q_2,p_3,p_4)\subset T_1$ and $\conv(q_1,p_3,p_4)\subset T_2$.

\end{proof}


\begin{lemma}
\label{lemma:3<4}
Let $P$ be a lattice parallelepiped and let $T\subset P$ be a tetrahedron. Then, at least one of the four corner tetrahedra $T_i$ of the circumscribed parallelepiped $C(T)$ is fully contained in $P$.
\end{lemma}

\begin{proof}
Let us denote the vertices of $T$ by $p_1, p_2, p_3, p_4$ and the vertices of $C(T)$ not in $T$ by $q_1, q_2, q_3, q_4$, with the conventions of \Cref{def:circunpara}. 

We call \emph{band} any region of the form $f^{-1}([\alpha,\beta])$ for some linear functional $f\in (\R^3)^*\setminus \{0\}$ and closed interval $[\alpha,\beta]\subset \R$.
We claim that any band containing $T$ must contain at least three of the $q_i$s. 
This claim implies that the parallelepiped $P$, which is the intersection of three bands, contains at least one of the $q_i$s and hence it fully contains the corresponding $T_i$.

To prove the claim, suppose that $q_1\not\in B:= f^{-1}([\alpha,\beta])$ for a certain band $B \supset T$. 
Without loss of generality, say $f(q_1)<\alpha$. Then the equalities $q_1+q_i=p_j+p_k$ and $q_1+p_1=q_i+p_i$, where $\{i,j,k\}=\{2,3,4\}$, respectively give:
\paco{used $i$ instead of $2$, to shorten proof}
\begin{gather}
\label{eq:first}
f(q_i) = f(p_j+p_k-q_1) = f(p_j)+f(p_k)-f(q_1) > 2\alpha-\alpha=\alpha,\\
\label{eq:second}
f(q_i) = f(q_1+p_1-p_2) = f(q_1)+f(p_1)-f(p_i) < \alpha + \beta-\alpha = \beta,
\end{gather}
so that $q_i \in B$ for $i\in \{2,3,4\}$.
%
%Inequality \eqref{eq:first} also implies
%\begin{equation}
%\label{eq:third}
%f(q_1) <  f(p_i) < f(q_2), \quad\text{ for $i=3,4$}.
%\end{equation}
%
%The translation by the vector $\frac12 (p_1+p_2-p_3-p_4)$ sends $q_1,q_2,p_3,p_4$ to $p_2,p_1,q_4,q_3$ (in this order). By applying this to inequality \eqref{eq:third}, we obtain
%\[
%\alpha \le f(p_2) < f(q_i) < f(p_1) \le \beta, \quad\text{ for $i=3,4$},
%\]
%so that $q_3,q_4 \in B$.
%
This concludes the proof of the claim, and of the lemma.

\end{proof}


\begin{corollary}
\label{coro:coverpara}
Let $T$ be an empty lattice tetrahedron contained in a lattice parallelepiped $P$. Then, $T$ can be covered by unimodular tetrahedra contained in $P$.
\end{corollary}
%\giulia{I would ignore comment Y.16, where if I understand correctly a different proof idea is suggested...}
\begin{proof}
We proceed by induction on the (normalized) volume of $T$, which is a positive integer. If this volume equals $1$ then $T$ is unimodular and there is nothing to prove, so we assume $T$ is not unimodular. Let $p_1, p_2, p_3, p_4$ denote the vertices of $T$.

\Cref{lemma:3<4} guarantees that one of the corner tetrahedra $T_i$ of the parallelepiped $C(T)$ is contained in $P$. Without loss of generality, suppose $T_4 = \conv(p_1, p_2, p_3,q_4)$ is in $P$. By \Cref{lemma:corner}, we know that $T_4$ contains a lattice point other than the $p_i$s, which we denote by $u$. 
%
Then $S=\conv(T\cup \{u\})$ can be triangulated in two different ways: $S=T \cup T'_4$, where $T'_4 = \conv(p_1, p_2, p_3, u) \subseteq T_4$ and $S= S_1 \cup S_2 \cup S_3$, with
\[
S_1= \conv(p_2,p_3,p_4, u),
S_2=\conv(p_1,p_3,p_4, u),
S_3=\conv(p_1,p_2,p_4, u).
\]

Each of the tetrahedra $S_i$ has lattice volume strictly smaller than $T$ because, for each $i$, $p_i$ is the unique point of $C(T)$ maximizing the distance to the opposite facet $\conv(p_j,p_k,p_l)$ of $T$. Thus, $S_1$, $S_2$ and $S_3$ cover $T$ and have volume strictly smaller than $T$. The $S_i$ may not be empty, but we can triangulate them into empty tetrahedra, which by inductive hypothesis can be covered unimodularly.
\end{proof}

\begin{proof}[Proof of \Cref{thm:parallelepipeds}]
Arbitrarily triangulate the parallelepiped into empty lattice tetrahedra and apply \Cref{coro:coverpara} to these tetrahedra.
\end{proof}


Let us say that a lattice $3$-polytope $P$ \emph{has the circumscribed parallelepiped property} if it satisfies the conclusion of \Cref{lemma:3<4}:  ``for every empty tetrahedron $T$ contained in $P$ at least one of the four corner tetrahedra in $C(T)$ is 
contained in $P$''. 
If this holds then $P$ has a unimodular cover, since then the proofs of \Cref{coro:coverpara} and \Cref{thm:parallelepipeds} work for $P$.
In turn, our proof that parallelepipeds have the property (\Cref{lemma:3<4}) is based on the fact that they have only three (pairs of) normal vectors. 
In the following two examples we show a smooth $3$-polytope and two $3$-polytopes with four normal vectors that do not have the property. The latter are not IDP:



\begin{example}[A smooth 3-polytope without the circumscribed parallelepiped property]
 Let $P$ be the Cayley embedding of a long horizontal rectangle and a long vertical rectangle. That is,
$P=\conv([0,a]\times [0,1] \times\{0\} \cup [0,1]\times [0,b] \times\{1\})$, for big $a$ and $b$. This is smooth and contains a big empty tetrahedron $T$ with vertices $(0,0,0)$, $(a,1,0)$, $(0,0,1)$, $(1,b,1)$ which occupies most of its volume. In particular, none of the corner tetrahedra of $T$ is contained in $P$. 

More explicitly, the the remaining vertices $q_i$ of the circumscribed parallelepiped are $(\frac{a+1}{2}, \frac{b+1}{2}, 1), (\frac{a+1}{2}, \frac{b+1}{2}, 0), (\frac{1-a}{2}, \frac{b-1}{2}, 1), (\frac{a-1}{2}, \frac{1-b}{2}, 1)$. None of these points are contained in $P$, and therefore none of the corner tetrahedra are either.
\giulia{Added the last part to answer comment Y.19}
\paco{edited it}
\end{example}


\begin{example}[Non-IDP polytopes with four facet directions]
\label{ex:non-IDP}
The following triangular prism $P$ and centrally symmetric octahedron $Q$ are not IDP:
\begin{gather}
P=\conv((0,1,1),(1,0,1),(1,1,0),(-1,0,0),(0,-1,0),(0,0,-1)),\\
Q= \conv((0,1,1),(1,0,1),(1,1,0),(0,-1,-1),(-1,0,-1),(-1,-1,0)).
\end{gather}

Indeed, in both cases the 
\paco{shortened}
point $(1,1,1)$ lies in the second dilation but is not the sum of two lattice points in the polytope. 
\end{example}

An affirmative answer to the following question (weaker than the circumscribed parallelepiped property) would still  imply that smooth $3$-polytopes can be unimodularly covered and, hence,~\Cref{conj:Oda}\eqref{itm:smoothIDP} in dimension three:

\begin{question}
If $T$ is an empty tetrahedron contained in a smooth 3-polytope $P$, can one guarantee that there is a lattice point of $P$ in the circumscribed parallelepiped of $T$ (apart of the vertices of $T$)?
\end{question}




\section{ Cayley sums}
\label{sec:cayley}
Let $P$ and $Q$ be two lattice polytopes in $\R^d$. We do not require them to be full-dimensional, but we assume their Minkowski sum is. Remember that the \emph{Minkowski sum} $P+Q$ and the \emph{Cayley sum} of $P$ and $Q$ are defined as:
\begin{gather*}
P + Q := \{ p+q \in \R^d: p\in P, q \in Q\} \subset \R^d,\\
\cayley (P,Q) = \conv( P\times\{0\} \cup Q\times \{1\}) \subset \R^{d+1}.
\end{gather*}


The so-called \emph{Cayley Trick} is the isomorphism
\[
2\cayley(P, Q) \cap (\R^d\times \{1\}) = (P+Q) \times \{1\} \cong P+Q,
\]
which easily implies:

\begin{proposition}[see, e.g.~\protect{\cite[Thm.~0.4]{Tsuchiya}}]
\label{prop:mixedIDP}
If $\cayley(P,Q)$ is IDP then the pair $(P,Q)$ is IDP.
\end{proposition} 
\giulia{We just called the pair IDP in the introduction... Which notation do we prefer?}
\paco{we stick to just IDP}

The Cayley Trick also provides the following canonical bijections:
\[
\begin{array}{ccc}
\text{polyhedral subdivisions of $\cayley(P,Q)$} &\leftrightarrow& \text{mixed subdivisions of $P + Q$}\\
\text{triangulations of $\cayley(P,Q)$} &\leftrightarrow& \text{fine mixed subdivisions of $P + Q$}\\
\text{unimodular simplices in $\cayley(P,Q)$} &\leftrightarrow& \text{unimodular prod-simplices in $P + Q$}.
\end{array}
\]
See \cite{DLRS2010} for more details on the Cayley Trick and on triangulations and polyhedral subdivisions of polytopes.
In fact these bijections can be taken as definitions of the objects in the right-hand sides. In particular, 
we call \emph{prod-simplices} in $P+Q$ the Minkowski sums $T_1+T_2$ where $T_1\subset P$ and $T_2\subset Q$ are simplices with complementary affine spans. A prod-simplex is \emph{unimodular} if the edge vectors from any vertex of $T_1$ and from any vertex of $T_2$ form a unimodular basis.

\medskip

We now turn our attention to $d=2$, in order to prove \Cref{thm:cayley}. A triangulation of $\cayley(P,Q)\subset \R^3$ consists of tetrahedra of types $(1,3)$, $(2,2)$ and $(3,1)$, where the type denotes how many vertices they have in $P$ and in $Q$. Empty tetrahedra of types $(1,3)$ or $(3,1)$, which are Cayley sums of an empty (hence unimodular) triangle in $P$ and a point in $Q$, or viceversa, are automatically unimodular. The case that we need to study are therefore tetrahedra of type $(2,2)$, which are Cayley sums of a segment $p\subset P$ and a segment $q\subset Q$. These correspond to prod-simplices of two segments in $P+Q$, which are \emph{parallelograms}.
The following lemma, whose proof we postpone to \Cref{sec:the_lemma}, is crucial to understand how to unimodularly cover these tetrahedra. 

We use the following conventions: if $a, b$ are points, we denote by $[a,b]$ and $(a,b)$ respectively the closed and open line segments with endpoints $a,b$. Given a segment $s=[a,b]$, we denote $\vec s$ the vector $b-a$ and denote $\vecline s$ the line spanned by $\vec s$. A lattice parallelogram $p+q$ is called \emph{unimodular} if it is a fundamental domain for the lattice. Equivalently, if $\cayley(p,q)$ is a unimodular tetrahedron.
\paco{added def of unimodular paralellogram}

\begin{lemma}
\label{lemma:cayley}
\paco{referee X suggests to give an ``example'' of the lemma, but I am not sure what could be helpful}
Let $Q\subset \R^2$ be a two-dimensional lattice polytope and $P\subset \R^2$ a weak Minkowski summand of it. 
Let $p=[p_1,p_2] \subset P$ and $q=[q_1,q_2]\subset Q$ be two primitive and non-parallel lattice segments, and let $\vecline p$ and $ \vecline q$ be the lines spanned by them.  If the parallelogram $p + q$ is not unimodular, then at least one of the regions
\[
((p_1, p_2) + \vecline q ) \cap P, 
\qquad \text{and} \qquad
((q_1, q_2) + \vecline p ) \cap Q
\]
contains a lattice point. 
\end{lemma}

See \Cref{fig:strips} for an illustration of the two regions in the statement, which we call \emph{strips}. In this figure and the forthcoming ones in \Cref{sec:the_lemma} 
we draw $p$ as a vertical segment and $q$ as a horizontal one for convenience. This is always possible via a linear transformation (which of course changes the lattice; in the proof we do not assume the lattice to be $\Z^2$). 
\paco{moved this remark here, since it already affects this proof}
%\Giulia{I'm not sure this last sentence addition is necessary, but I thought it would help the ref clear up confusion (Y.31) about p being a vertical segment-sure, this isn't possible with a unimod transformation, but we don't need to fix the lattice...}


\begin{figure}[htb]
\scalebox{.75}{\input{strips.pdf_t}}
\caption{The strips  of Lemma \ref{lemma:cayley}}
\label{fig:strips}
\end{figure}

\begin{corollary}
\label{coro:covercayley}
Let $T$ be an empty lattice tetrahedron contained in the Cayley sum $\cayley(P,Q)$, where $Q$ is a lattice polygon and $P$ is a weak Minkowski summand of $Q$. Then, $T$ can be covered by unimodular tetrahedra contained in $\cayley(P,Q)$.
\end{corollary}

\begin{proof}
The proof is by induction on the normalized volume of $T$, which we assume to be at least $2$. This implies that $T$ is of type $(2,2)$,
since empty tetrahedra of types $(1,3)$ and $(3,1)$ are unimodular. Thus, $T$ is the Cayley sum of primitive segments $p=[p_1,p_2]\subset P$ and $q=[q_1,q_2]\subset Q$.  
Let $u$ be the lattice point whose existence is guaranteed by \Cref{lemma:cayley}. Assume  (the other case is similar) that 
\[
u \in ((p_1, p_2) + \vecline q ) \cap P,
\]
and call $t$ the triangle $t=\conv( u,  p_1,  p_2)\subset P$.

Let us denote $\tilde u$, $\tilde p_1$, $\tilde p_2$, $\tilde q_1$, $\tilde q_2$ the points corresponding to $u, p_1, p_2, q_1, q_2$ in $\cayley(P,Q)$.
That is, $\tilde p_i = p_i\times\{1\}$, $\tilde q_i = 	q_i\times\{0\}$, and $\tilde u = u\times\{1\}$.\giulia{to match the figure, P is now at height 1 instead of 0}
Observe that the assumption $u\in((p_1, p_2) + \vecline q$ implies that one of the segments $[u,q_i]$ crosses the interior of one of the triangles $\conv(p_1,p_2,q_j)$, where $\{i,j\}=\{1,2\}$. Without loss of generality assume that $[\tilde{u},\tilde{q}_2]$ crosses $\conv(\tilde{p}_1,\tilde{p}_2,\tilde{q}_1)$, as in \Cref{fig:flip}. 

\begin{figure}[htb]
\includegraphics[scale=.3]{flip}
\caption{$[\tilde{u},\tilde{q}_2]$ intersects $\conv(\tilde{p}_1,\tilde{p}_2,\tilde{q}_1)$}
\label{fig:flip}
\end{figure}\giulia{added tildes to the points}

In turn, this means that the polytope $\conv(\tilde u, \tilde p_1, \tilde p_2, \tilde q_1, \tilde q_2) = \cayley(t,q)$
 has the following two triangulations:
\begin{gather*}
\mathcal T^+:= \left\{ \cayley(p,q), \cayley(t, \{q_2\}) \right\},
\\
\mathcal T^-:= \{ \cayley([p_1,u],q), \cayley([p_2,u],q), \cayley(t, \{q_1\}) \}.
\end{gather*}
The tetrahedra $\cayley(t, \{q_1\})$ and $\cayley(t, \{q_2\})$ are unimodular, which implies that $T=\cayley(p,q)$ has volume equal to the sum of the volumes of $\cayley([p_1,u],q)$ and $\cayley([p_2,u],q)$. In particular, we have covered $T$ by the three tetrahedra in $\mathcal T^-$, which are of smaller volume and hence have unimodular covers by induction hypothesis.
\end{proof}

\begin{proof}[Proof of \Cref{thm:cayley}]
Arbitrarily triangulate $\cayley(P,Q)$ into empty lattice tetrahedra and apply \Cref{coro:covercayley} to these tetrahedra.
\end{proof}

Let us now show how to derive \Cref{coro:prismatoid} from this theorem.
\emph{Prismatoids} were defined in~\cite{Santos-hirsch} as polytopes whose vertices all lie in two parallel facets. In particular, a \emph{lattice prismatoid} is any $d$-polytope $SL(\Z,d)$-equivalent to one of the form
\[
\conv(Q_1\times\{0\} \cup Q_2 \times \{k\}),
\]
where $Q_1,Q_2$ are lattice $(d-1)$-polytopes and $k\in \Z_{>0}$. This is almost a generalization of Cayley sums, which would be the case $k=1$, except the definition of prismatoid requires $Q_1$ and $Q_2$ to be full-dimensional, while the Cayley sum only requires this for $Q_1+Q_2$.

\begin{proposition}
\label{prop:prismatoid}
Let $Q_1$, $Q_2$ be two lattice polygons and consider the prismatoid 
\[
P:= \conv(Q_1\times\{0\} \cup Q_2 \times \{k\},
\]
with $k\ge 2$. 
%Assume that $P\cap(\R^2\times\{i\})$ is a lattice polytope for every $i=0,\dots, k$. This is equivalent to assuming that $P\cap(\R^2\times\{1\})$ is a lattice polytope:
If $P\cap(\R^2\times\{1\})$ is a lattice polygon then $P$ has a unimodular cover.
\end{proposition}

\begin{proof}
The condition that $P\cap(\R^2\times\{1\})$ is a lattice polygon implies the same for $P\cap(\R^2\times\{i\})$, for every $i$. 
Indeed, the condition implies that every edge of $\cayley(P,Q)$ of the form $[u\times \{0\}, v\times \{k\}]$ has a lattice point in $\R^2\times\{i\}$, and hence it has a lattice point in $P\cap(\R^2\times\{i\})$, for every $i$.

Observe that for every $i\in \{1,\dots,k-1\}$ the intersection $P\cap(\R^2\times\{i\})$ has the same normal fan as $Q_1+Q_2$. Thus, each slice
\[
P \cap (\R^2\times[i-1,i])
\]
is a Cayley polytope. For $i\in\{2,\dots,k-1\}$,  both bases have the same normal fan (and therefore each is a weak Minkowski summand of the other); for $i\in \{1,k\}$ one base is a weak Minkowski summand of the other. We can therefore apply \Cref{thm:cayley} to each slice and combine the covers thus obtained to get a unimodular cover of $P$.
\end{proof}


\begin{proof}[Proof of \Cref{coro:prismatoid}]
The polytope under study satisfies the hypotheses of \Cref{prop:prismatoid}: %in \Cref{coro:prismatoid}, 
the smoothness of the prismatoid implies that every edge of the form $[u\times \{0\}, v\times \{k\}]$ has lattice points in all slices. 
Hence,
\[
kP \cap(\R^2\times\{1\}) = (k-1)Q_1 + Q_2.
\qedhere
\] 
\end{proof}




\section{Proof of \Cref{lemma:cayley}}
\label{sec:the_lemma}
Let $f_q$ be the primitive lattice functional constant on $q$ and $f_p$ the one constant on $p$. We assume that $f_q(p_1) < f_q(p_2)$ and $f_p(q_1) < f_p(q_2)$.

\paco{rphrased to address X.10}
Since we can perform without loss of generality respective lattice translations to $P$ and to $Q$, we assume that $q_1$ is the origin and that $f_q(p_1)=-1$. 
Observe that then $f_q(p_2)$ must be strictly positive, since $f_q(p_2)=0$ would imply that $p+q$ is a unimodular paralellogram. Moreover, since $p$ is primitive $p_1$ and $p_2$ cannot be in the boundary of $q +\vecline p$, which implies that $p_1$ is the unique lattice point with $f_q(x)=-1$ in $q +\vecline p$. Similarly, the unique lattice point in the strip with $f_q(x)=1$ is  $q_1+q_2 -p_1$.

%
%;  indeed, since $q$ is primitive, the only way that in the strip there could be two lattice points on $f_q(x)=-1$ is if they were on the boundary of the strip, which would however imply that $p+q$ is a unimodular paralellogram, against our assumptions.
%Since translating the polytopes independently by lattice vectors will not result in any loss of generality, we can assume that $p_1$ is that unique lattice point. That is, $f_q(p_1)=-1$, or equivalently, the triangle $\conv(q_1, q_2, p_1)$ is unimodular. Similarly, the unique lattice point in the strip on the line $f_q(x)=1$ is then $q_1+q_2 -p_1$.

We let $H_1=\{f_q(x) \leq 0\}$ and $H_2=\{f_q(x) \geq 0\}$; similarly let $V_1=\{f_p(x) \leq f_p(p)\}$ and $V_2=\{f_p(x) \geq f_p(p)\}$.%
\paco{added footnote}
\footnote{Since in our figures $p$ and $q$ are vertical and horizontal, we use the letters $V$ and $H$ for the half-planes they defined. Similarly, later in the proof we use the letters $b$, $t$, $r$, and $l$ for certain  points and segments meaning ``bottom'', ``top'', ``right'' and ``left''.}
%
\begin{figure}[htb]
\includegraphics[scale=.3]{setup.png}
\caption{Setup for the proof of \Cref{lemma:cayley}}
\label{fig:setup}
\end{figure}

Let $w=\area(p+q) \geq 2$, where $\area$ denotes the area normalized to a fundamental domain. In what follows, the width of a functional $f$ on a set $S$, denoted $\width_f(S)$ is defined as the difference $\sup_{x\in S} f(x)-\inf_{x\in S} f(x)$.
Then:
\[
w=\width_{f_q}(p + \vecline q )=\width_{f_q}(p)=\width_{f_p}(q)=\width_{f_p}(q +\vecline p ). 
\] 


\begin{proof}[Proof of \Cref{lemma:cayley}]
Suppose by contradiction that there is no lattice point as described in the lemma. In particular, no lattice point on the boundary of $Q$ can be in the interior of the strip $q + \vecline p$.  Thus the boundary of $Q$ contains two primitive segments which each have one vertex on each side of the strip $q + \vecline p$; we will call these  $b=[b_1, b_2], t=[t_1, t_2]$, with $b$ and $t$ crossing the strip in $H_1$ and $H_2$ respectively and the convention that $f_p(b_2) >f_p(b_1)$ and $f_p(t_2) >f_p(t_1)$. This readily implies 
\begin{gather}
\label{eq:widthq}
\begin{array}{cc}
f_p(t_1) \leq f_p(q_1), &
f_p(t_2) \geq f_p(q_2), \\
f_p(b_1) \leq f_p(q_1), &
f_p(b_2) \geq f_p(q_2).
\end{array}
\end{gather}


The same holds for $P$ and the strip $p+\vecline q$, and we call the segments 
\paco{not sure why we use $\ell$ for the segment and $l$ for the endpoints...}
$\ell=[l_1, l_2]$ and $r=[r_1, r_2]$, with $\ell$ and $r$ crossing the strip $p + \vecline q$  in $V_1$ and $ V_2$ respectively. The only difference is that in the case that $P$ is one dimensional we have $\ell=r=p$.  Again we have
\begin{gather}
\label{eq:widthp}
\begin{array}{cc}
f_q(l_1) \leq f_q(p_1), &
f_q(l_2) \geq f_q(p_2), \\
f_q(r_1) \leq f_q(p_1),&
f_q(r_2) \geq f_q(p_2).
\end{array}
\end{gather}

Observe that a priori one of $\ell$ and $r$ can coincide with $p$, if $p$ is on the boundary of $P$, and similarly one of $t,b$ might be $q$, if $q$ is on the boundary of $Q$. 


\begin{claim}\label{claim:width}
The following inequalities hold, 
\begin{align*}
\width_{f_q}(\ell) , 
\width_{f_q}(r) ,
\width_{f_p}(t) ,
\width_{f_p}(b) \geq w.
\end{align*}
Each inequality is strict, unless the segment in question coincides with $p$ or $q$.
\end{claim}

\begin{proof}
The inequality $\geq w$ follows in each case from \eqref{eq:widthp} and \eqref{eq:widthq}.

If one of the inequalities, say the one for $\ell$, is not strict, then $\ell$ has one endpoint on each of the boundary lines of $(p + \vecline q)$. Unless $\ell = p$, one of the endpoints of $\ell$ is not an endpoint of $p$, say $l_1 \neq p_1$. Thus the triangle $T=\conv(p_2, p_1, l_1)$ is contained in $P$ and its edge $[p_1, l_1]$ is an integer dilation of $q$. 

Since $T$ contains $p$ and a copy of $q$, its area (normalized to a fundamental domain) is $w/2 \geq 1$, and by Pick's theorem it must contain a lattice point other than its vertices. Since $p$ and $q$ are primitive, this lattice point must lie in the interior of the strip.\giulia{added an explanation here, according to the suggestion Y.32}
%Since $\width_{f_q}(T) =w \geq 2$, $T$ must contain a lattice point in the interior of the strip.
\end{proof}


\begin{claim}
\label{claim:b_and_t}
$f_q(b_2-b_1)$ and $f_q(t_2 - t_1)$ are non-zero and have the same sign. That is, $f_q$ achieves its maximum over $b$ and over $t$ on the same halfplane $V_1$ or $V_2$.
\end{claim}

\begin{proof}
Both $t$ and $b$ must cross the interior of $p$, or otherwise $p_1$ or $p_2$ are the lattice points we are looking for in $Q$.
\paco{simplified proof}
To seek a contradiction assume, as in \Cref{fig:claim2}, that 
\[
f_q(t_2 - t_1) \le 0 \le f_q(b_2-b_1).
\]
That is, $f_q$ decreases (perhaps weakly) along $t$ and increases along $b$, as $f_p$ increases on both. This implies that $Q \cap V_2$ is contained in the open strip $\{f_q(p_1) <f_q(x)< f_q(p_2)\}$, of width $w$.
%
%
%Then, $f_q(t_2) \le  f_q (t_1)$ implies $f_q(t_2)$ equals the minimum of $f_q$ on $t$, which is smaller than $f_q(p_2)=w-1$. Similarly, 
%
% f_q(t_2)
%
%This implies 
%
%
%
%Let us first show that $f_q(b_2-b_1)\ne 0 \ne f_q(t_2 - t_1)$. If, say, $f_q(b_2-b_1)= 0$, then $b$ is parallel to $q$ and its endpoints $b_1$ and $b_2$ must be on the boundary of the strip. By \Cref{claim:width} this implies $b=q$.
%
%Then, $[b_i,q_i]$ is parallel to $p$, for $i=1,2$.
%
%The left (or right) endpoint of this translate and of $q$ are lattice points lying on a line parallel to $p$, and since both this translate of $q$ and $q$ cross the segment $p$, they force $p$ to contain a lattice point in the interior, a contradiction to $p$ being primitive.  
%	
%Suppose by contradiction that the maximum of $f_q$ on $t$  lies in $V_1$ and that the maximum on $b$ lies in $V_2$. %
%Then $Q \cap V_2$ is contained in the open strip $\{-1<f_q(x)<w-1\}$, of width $w$. 
This strip cannot contain a translated copy of $r$, since $\width_{f_q}(r) \geq w$, which gives a contradiction: since $P$ is a weak Minkowski summand of $Q$,  $Q$ must have edges parallel to $r$ both in $V_1$ and $V_2$.
\end{proof}

\begin{figure}[htb]
\scalebox{.75}{\input{claim2.pdf_t}}
\caption{Illustration of the proof of \Cref{claim:b_and_t}}
\label{fig:claim2}
\end{figure}

We assume without loss of generality that the maximum on $t$ (and hence on $b$) is achieved in $V_2$, that is to say, $f_p$ and $f_q$ increase in the same direction along $t$ (and hence along $b$). Otherwise the following considerations can be applied to $V_1$. 

\begin{claim}
\label{claim:r}
Assume without loss of generality that $b$ and $t$ either are parallel or their affine spans cross in $V_2$ (if they cross in $V_1$, the same claim can be reworded for $V_1$ and $l$). Then, 
\begin{enumerate}
\item The intersection of $Q$ with any line parallel to $p$ in $V_2$ has width with respect to $f_q$ strictly smaller than $w$.
\item $f_p(r_2) > f_p(r_1)$, that is, $f_p$ achieves its maximum over $r$ in $H_2$.
\end{enumerate}
\end{claim}

\begin{proof}
Both $t$ and $b$ must intersect $p$, as said in the proof of \Cref{claim:b_and_t}.  Their intersections with $p$ are thus endpoints of a segment of width with respect to $f_q$ less than $w$, the width of $p$. Since $t$ and $b$ cross in $V_2$, the same is true for any segment parallel to $p$ contained in $Q \cap V_2$.  

\begin{figure}[htb]
\scalebox{.75}{\input{claim3.pdf_t}}
\caption{Illustration of the proof of \Cref{claim:r}}
\label{fig:claim3}
\end{figure}

For part (2), recall that by \Cref{claim:width}, $\width_{f_q}(r) \geq w$. If $f_p(r_2) \leq f_p(r_1)$, it would be impossible to fit a translated copy $r'$ of $r$ in the correct side of $Q$: since $f_q$ increases along $t$, $r'$ has width in direction $f_q$ smaller than the segment parallel to $p$ with endpoint $r'_1$. This segment by part (1) has width less than $w$ in direction $q$, a contradiction. \giulia{simplified the proof as suggested in Y.38} 

%$r'$ would need to lie inside the triangle delimited by the affine line $\langle t \rangle$ and the inequalities $f_q(x) \geq f_q(r_1)$, $f_p(x) \leq f_p(r_1)$. However, this triangular region has width less than $w$  with respect to $f_q$, by combining part (1) with the fact that $f_p$ and $f_q$ increase in the same direction along $t$, see \Cref{fig:claim3}. 
\end{proof}

The last two claims can be summarized as saying that in the pictures $b$, $t$ and $r$ have positive slope. Observe that this implies that $q$ is not in the boundary of $Q$ and $p \neq r$, so both $P$ and $Q$ are full dimensional.


Let $g$ be the primitive lattice functional constant on $[p_1, q_2]$ (and therefore constant also on $[q_1, q_1+q_2-p_1]$). By the assumption on $p_1$, the values of $g$ on these segments differ by $1$. We choose the sign of $g$ so that 
\[
g([p_1, q_2])= g( [q_1, q_1+q_2-p_1]) -1. 
\]
\giulia{changed the +1 to a -1, so that the inequalities in the following claim are the right way around--we had flipped everything}

\begin{claim}
\label{claim:g}
$g(t_1) > g(t_2)$, $g(b_1) > g(b_2)$, and $g(r_1) < g(r_2)$.
\end{claim}

\begin{proof}
Since $b$ and $t$ must respectively separate $p_1$ and $q_1+q_2-p_1$ from the other two vertices of the parallelogram $\conv(q_1, p_1, q_2, q_1+q_2-p_1)$, they must respectively intersect its (parallel) edges $[p_1, q_2]$ and $[q_1, q_1+q_2-p_1]$, which implies the stated inequalities for $b$ and $t$.
The same argument  applied to the parallelogram  $\conv(p_1, q_2, p_2, p_1+p_2-q_2)$, yields the inequalities for $r$.
\end{proof}

\begin{figure}[htb]
\scalebox{.75}{\input{claim4.pdf_t}}
\caption{Illustration of the proof of \Cref{claim:g}}
\label{fig:claim4}
\end{figure}


We are now ready to show a contradiction. Since the normal fan of $Q$ refines that of $P$, $Q$ must have an edge $r'$ which is a translated copy of $r$. Let $r_1'$ and $r_2'$ be its endpoints. 

\Giulia{Changed the proof from here}
Consider the segment $s$ contained in $r_1' + \vecline p$ with endpoints $s_1=r_1'$ on $d$ and $s_2$ on the line spanned by $t$. The width of $s$ with respect to $g$ is $g(s_2)-g(s_1) <1$, because $s$ is shorter than $Q \cap p$, which is strictly contained between the consecutive parallel lattice lines through $p_1$ and $q_2$ and through $q_1$ and $q_1+q_2-p_1$ where $g$ is constant.

We now observe that $g(r_2) < g(s_2)$, because $r_2$ lies to the right of $s_2$ ($r$ has positive slope) and below the line spanned by $t$, and $g$ decreases moving to the right along $t$ (\Cref{claim:g}). Thus $g(r_2)-g(r_1) = g(r'_2)-g(r'_1) < g(s_2)-g(s_1) <1$, that is, $g(r_2)=g(r_1)$. This contradicts \Cref{claim:g}.
%Since $t$ separates $q_1$ and $q_1+q_2-p_1$ and $g$ decreases from $t_1$ to $t_2$ (by \Cref{claim:g}), the inequality $g(x)< g(d')$ holds on $Q\cap V_2$, and in particular for $r_2'$. Since $r_2'$ is a lattice point, $g(r_2')\leq g(d)= g(r_1')$, which contradicts \Cref{claim:g}).
\end{proof}


\begin{thebibliography}{99} 

\bibitem{BHHHJKM2019}
Matthias Beck, Christian Haase, Akihiro Higashitani, Johannes Hofscheier, Katharina Jochemko, Lukas Katth\"an, Mateusz Micha{\l}ek,
Smooth Centrally Symmetric Polytopes in Dimension 3 are IDP,
\emph{Ann.~Combin.}, 23 (2019), 255--262.
\doi{10.1007/s00026-019-00418-x}


\bibitem{BG}
Winfried Bruns and Joseph Gubeladze, Normality and covering properties of affine semigroups, \emph{J. Reine Angew. Math.} 510 (1999), 151--178. \doi{10.1515/crll.1999.044}

\bibitem{BGbook}
Winfried Bruns and Joseph Gubeladze, \emph{Polytopes, rings, and k-theory}, Monographs in Mathematics,
  Springer-Verlag, 2009, XIV, 461 p. 52 illus.
  
\bibitem{BGM}
  Winfried Bruns, Joseph Gubeladze, and Mateusz Michalek. Quantum Jumps of Normal Polytopes. \emph{Discrete Comput. Geom.}, 56(1) (2016), 181--215. \doi{10.1007/s00454-016-9773-7}
  
\bibitem{CLS}
David A.~Cox, John B.~Little, and Hal K.~Schenck.
{\em {T}oric {V}arieties}.
AMS, Providence, 2011.

 \bibitem{DLRS2010}
J. A. De Loera, J. Rambau, F. Santos
\emph{Triangulations: Structures for Algorithms and Applications}, 539 pp.
Algorithms and Computation in Mathematics, Vol. 25, Springer-Verlag. 
ISBN: 978-3-642-12970-4

\bibitem{Fakhruddin}
Najmuddin Fakhruddin.
\newblock Multiplication maps of linear systems on smooth projective toric surfaces.
  Preprint, 
  \href{https://arxiv.org/abs/math/0208178}{arXiv:math/0208178}, August 2002.


%\bibitem{Grunbaum}
%{Branko Gr\"unbaum}.
%{\em {C}onvex {P}olytopes}.
%Wiley, London, 1967.

\bibitem{Gubeladze}
Joseph Gubeladze,
Normal polytopes and ellipsoids,
\emph{Elect. J. Combin.}, 28:4 (2021), P4.7.
\doi{10.37236/10338}
    

\bibitem{HaaseHof}
Christian Haase and Jan Hofmann, 
Convex-normal (Pairs of) Polytopes, 
\emph{Canad. Math. Bull.} 60:3 (2017), 510--521. \doi{10.4153/CMB-2016-057-0}

  \bibitem{HPPS-survey}
Christian Haase, Andreas Paffenholz, Lindsay C. Piechnik, Francisco Santos. Existence of unimodular triangulations - positive results. 
 \emph{Mem. Amer. Math. Soc.},  Vol.~270 1N.~321, American Math. Society, 2021. DOI: \doi{10.1090/memo/1321}
%ISBNs: 978-1-4704-4716-8 (print); 978-1-4704-6530-8 (online)
 
\bibitem{HNPS2008}
Christian Haase, Benjamin Nill, Andreas Paffenholz, and Francisco Santos, Lattice points in Minkowski sums, 
\emph{Electron.~J.~Combin.}, 15:1 (2008), Note 11, 5 pp. \doi{10.37236/886}


\bibitem{KantorSarkaria}
Jean-Michel Kantor and Karanbir~S.\ Sarkaria.
On primitive subdivisions of an elementary tetrahedron, 
\emph{Pacific J.~Math.} \textbf{211} (2003), no.~1,  123--155. \doi{10.2140/pjm.2003.211.123}

\bibitem{mfo2004}
Mini-workshop: Ehrhart Quasipolynomials: Algebra, Combinatorics, and Geometry, Oberwolfach Rep. 1 (2004), no. 3, 2071-2101, Abstracts from the mini-workshop held August 15-21, 2004, Organized by Jes\'us A.~De Loera and Christian Haase, Oberwolfach Reports. Vol. 1, no. 3. \doi{10.14760/OWR-2004-39}

\bibitem{mfo2007}
Mini-workshop: Projective normality of smooth toric varieties, Oberwolfach Rep. 4 (2007), no. 39/2007, Abstracts from the mini-workshop held August 12-18, 2007. Organized by Christian Haase, Takayuki Hibi and Diane MacLagan. \doi{10.14760/OWR-2007-39}

\bibitem{Oda1997}
Tadao Oda. Problems on Minkowski sums of convex lattice polytopes. 
Abstract submitted at the Oberwolfach Conference ''Combinatorial Convexity and Algebraic Geometry'' 26.10--01.11, 1997.
Available as arXiv preprint \href{https://arxiv.org/abs/0812.1418}{arXiv:0812.1418}, 2008.

\bibitem{Ogata}
Shoetsu Ogata.
\newblock {Multiplication maps of complete linear systems on projective toric surfaces.}
\newblock {\em Interdiscip. Inf. Sci.}, 12(2):93--107, 2006. \doi{10.4036/iis.2006.93}

\bibitem{Santos-hirsch}
Francisco Santos.
A counter-example to the Hirsch Conjecture.
\emph{Ann.~Math.~(2)}, 176 (July 2012), 383--412. 
\doi{10.4007/annals.2012.176.1.7}

\bibitem{SantosZiegler}
Francisco Santos and G{\"u}nter~M. Ziegler.
\newblock {Unimodular triangulations of dilated 3-polytopes}, 
\newblock {\em Trans. Moscow Math. Soc.} (2013), 293--311. \doi{10.1007/s00493-014-2959-9}

\bibitem{Tsuchiya}
Akiyoshi Tsuchiya.
Cayley sums and Minkowski sums of 2-convex-normal lattice polytopes.
Preprint,  \href{https://arxiv.org/abs/1804.10538}{arXiv:1804.10538}, April 2018.


\bibitem{White1964}
G.~K.~White.
Lattice tetrahedra.
\emph{Canadian J.~Math.} 16 (1964), 389--396. \doi{10.4153/CJM-1964-040-2}

\end{thebibliography}


\end{document}
