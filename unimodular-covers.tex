\documentclass{amsart}

\textheight=22.5cm
\topmargin=0.5cm

\usepackage{amssymb}
\usepackage{graphicx,color}
\usepackage{amsmath}
\usepackage{url}
\usepackage[colorlinks=true]{hyperref}
%\usepackage{pictex}



\usepackage{cleveref}
\Crefname{counterexample}{counterexample}{counterexamples}
\Crefname{counterexample}{Counterexample}{Counterexamples}
\Crefname{conjecture}{conjecture}{conjectures}
\Crefname{conjecture}{Conjecture}{Conjectures}


\usepackage{tikz-cd}
\usepackage{tikz}
\usetikzlibrary{matrix}
\usepackage{tkz-euclide}
\usetikzlibrary{intersections}
\usetkzobj{all}


\newcommand{\limites}{}

\theoremstyle{plain}
\newtheorem{theorem}{Theorem}[section]
\newtheorem{proposition}[theorem]{Proposition}
\newtheorem{lemma}[theorem]{Lemma}
\newtheorem{corollary}[theorem]{Corollary}
\newtheorem{claim}[theorem]{Claim}

\theoremstyle{definition}
\newtheorem{definition}[theorem]{Definition}
\newtheorem{notation}[theorem]{Notation}
\newtheorem{example}[theorem]{Example}
\newtheorem{problem}[theorem]{Problem}
\newtheorem{question}[theorem]{Question}
\newtheorem{conjecture}[theorem]{Conjecture}
\newtheorem{remark}[theorem]{Remark}


% Macros in general

\newcommand{\F}{ \ensuremath{\mathbb{F}}}
\newcommand{\N}{ \ensuremath{\mathbb{N}}}
\newcommand{\Z}{ \ensuremath{\mathbb{Z}}}
\newcommand{\C}{ \ensuremath{\mathbb{C}}}
\newcommand{\R}{ \ensuremath{\mathbb{R}}}

\newcommand{\T}{ \ensuremath{\mathcal{T}}}

\newcommand{\bb}{{\mathbf{b}}}
\newcommand{\cc}{{\mathbf{c}}}
\newcommand{\ttt}{\mathbf{t}}
\newcommand{\GL}{{GL}_r (K)}

\renewcommand{\int}{\operatorname{int}}
\newcommand{\width}{\operatorname{width}}
\newcommand{\ind}{\operatorname{index}}
\newcommand{\Vol}{\operatorname{Vol}}
\newcommand{\area}{\operatorname{area}}
\newcommand{\length}{\operatorname{length}}

\renewcommand{\vec}[1]{\overrightarrow#1}
\newcommand{\vecline}[1]{\langle \vec #1 \rangle}


\newcommand{\cone}{\ensuremath{\mathrm{cone}}\hspace{1pt}}
\newcommand{\conv}{\ensuremath{\mathrm{conv}}\hspace{1pt}}
\newcommand{\cayley}{\operatorname{Cay}}
\newcommand{\rank}{\operatorname{rank}}
\newcommand{\reg}{\operatorname{reg}}

\newcommand{\Image}{\ensuremath{\mathrm{Im}}\hspace{1pt}}

\usepackage[colorinlistoftodos]{todonotes}

\newcommand{\giulia}[1]{\todo[size=\tiny,color=blue!30]{#1 \\ \hfill --- G.}}
\newcommand{\Giulia}[1]{\todo[size=\tiny,inline,color=blue!30]{#1 \\ \hfill --- G.}}

\newcommand{\paco}[1]{\todo[size=\tiny,color=green!30]{#1 \\ \hfill --- P.}}
\newcommand{\Paco}[1]{\todo[size=\tiny,inline,color=green!30]{#1 \\ \hfill --- P.}}



\date{\today}

\author[G.~Codenotti]{Giulia Codenotti}
\author[F.~Santos]{Francisco Santos}

\address[G.~Codenotti]
{
Institut f\"ur Mathematik, Freie Universit\"at Berlin, Germany
}
\email{giulia.codenotti@fu-berlin.de}

\address[F.~Santos]
{
Department of Mathematics, Statistics and Computer Science, University of Cantabria, Spain
}
\email{francisco.santos@unican.es}

\thanks{The authors were supported by the Einstein Foundation Berlin under grant EVF-2015-230. 
Work of F. Santos is also supported by project MTM2017-83750-P of the Spanish Ministry of Science (AEI/FEDER, UE)}

%%%%%%%%%%%%%%%%%%%%%%%%%%%%%%%%%%%%%%%%%%%%%%%%%%%%%%%%%%%

\title{Unimodular covers of 3-dimensional parallelepipeds and Cayley sums}

\begin{document}

\begin{abstract}
We show that the following classes of three-dimensional lattice polytopes have unimodular covers: the class of all parallelepipeds, the class of all centrally symmetric polytopes, and the class of all three-dimensional Cayley sums $\cayley(P,Q)$ where the normal fan of $Q$ refines that of $P$. This improves results of Beck et al.~(2018) and Haase et al.~(2008) where the last two classes were shown to be IDP.
\end{abstract}

\maketitle

\tableofcontents

\section{Introduction}

A lattice polytope $P\subset \R^d$ has the \emph{integer decomposition property} if for every positive integer $n$, every lattice point $p \in nP\cap \Z^d$ can be written as a sum of $n$ lattice points in $P$. We abbreviate this by saying that ``$P$ is IDP''. Being IDP is interesting in the context of  both enumerative combinatorics (Ehrhart theory) and algebraic geometry (normality of toric varieties). It falls into a hierarchy of several properties each stronger than the previous one; see, e.g., \cite[Section 2.D]{BGbook}, \cite[Sect. 1.2.5]{HPPS-survey}, \cite[p. 2097]{mfo2004}, \cite[p. 2313]{mfo2007}.
Let us here only mention that
\[
P \text{ has a unimodular triangulation}\Rightarrow
P \text{ has a unimodular cover}\Rightarrow
P \text{ is IDP.}
\]
Remember that a \emph{unimodular triangulation} is a triangulation of $P$ into unimodular simplices, and a \emph{unimodular cover} is a collection of unimodular simplices whose union equals $P$. 

Oda (\cite{Oda1997}) posed several questions regarding smoothness and the IDP property for lattice polytopes.
%
\paco{I think this definition goes better here, and decided to "undeclare" it and the definition of weak Mink sum}
Following \cite{HaaseHof, Tsuchiya}, we say that a pair $(P, Q)$ of lattice polytopes has the integer decomposition property, or that \emph{the pair $(P,Q)$ is IDP}, if 
\begin{align*}
\label{eq:mixedIDP}
(P+Q) \cap \Z^d = P \cap \Z^d + Q \cap \Z^d.
\end{align*}
%
A lattice polytope $Q$ is called \emph{smooth} if it is simple and the primitive edge directions \giulia{changed this from "edge generators", which to me was unclear} at every vertex form a linear basis for the lattice; equivalently, if the projective toric variety defined by the normal fan of $Q$ is smooth. 
The following versions of Oda's questions are now considered conjectures~\cite{HNPS2008,mfo2007}, and they are open even in dimension three:
\begin{conjecture}
\label{conj:Oda}
\begin{enumerate}
\item 
\label{itm:smoothIDP}
(Related to problems 2 and 5 in \cite{Oda1997})
Every smooth lattice polytope is IDP.
\item 
\label{itm:mixedIDP}
(Related to problems 1, 3, 4, 6 in \cite{Oda1997}) Every pair $(P,Q)$ of lattice polytopes with $Q$ smooth and the normal fan of $Q$ refining that of $P$ is IDP.
%\begin{align}
%\label{eq:mixedIDP}
%(P+Q) \cap \Z^d = P \cap \Z^d + Q \cap \Z^d.
%\end{align}
\end{enumerate}
\end{conjecture}

When the normal fan of a $Q$ refines that of $P$, as in the second conjecture, we say that $P$ \emph{is a weak Minkowski summand of $Q$}, since this is equivalent to the existence of a polytope $P'$ such that $P+P' = k Q$ for some dilation constant $k>0$\paco{did not find this in the google books version of \cite{Grunbaum}; found it in Postnikov-Reiner-Williams, Tsuchiya, a survey by Rekha Thomas, and other places. Perhaps we can leave it wothout reference}. 
This property has the following algebraic implication for the projective toric variety $X_Q$: $P$ is a weak Minkowski summand of $Q$ if and only if the Cartier divisor on  $X_Q$ defined by $P$ is \emph{numerically effective}, or ``nef'' (see~\cite[Cor.~6.2.15, Prop.~6.3.12]{CLS}, but observe that there, what we call  ``weak Minkowski summand'' is called simply ``Minkowski summand''). 
%\paco{added this explanation}
%
%\begin{definition}
%Following \cite{HaaseHof, Tsuchiya} we say that the pair $(P, Q)$ has the integer decomposition property, or \emph{the pair $(P,Q)$ is IDP}, if 
%\begin{align*}
%(P+Q) \cap \Z^d = P \cap \Z^d + Q \cap \Z^d.
%\end{align*}
%\end{definition}
% 
%In this language,  \Cref{conj:Oda}\eqref{itm:mixedIDP} can be rephrased as: if $P$ is a weak Minkowski summand of a smooth polytope $Q$ then the pair $(P,Q)$ is IDP.

\medskip 
Motivated by these and other questions, several authors have studied the IDP property for different classes of lattice polytopes. 
%
For example,  very recently
Beck et al.~\cite{BHHHJKM2019} proved that all smooth centrally symmetric $3$-polytopes are IDP.
More precisely, they show that any such polytope can be covered by lattice 
parallelepipeds and unimodular simplices, both of which are trivially IDP.
%
In \Cref{sec:parallelepipeds} we show:

\begin{theorem}
\label{thm:parallelepipeds}
Every $3$-dimensional lattice parallelepiped has a unimodular cover.
\end{theorem}

This, together with the mentioned result from~\cite{BHHHJKM2019}, gives:

\begin{corollary}
\label{coro:3cs}
Every smooth centrally symmetric lattice $3$-polytope has a unimodular cover. 
\qed
\end{corollary}

These results leave open the following important questions:

\begin{question}
Do $3$-dimensional parallelepipeds have unimodular triangulations?
\end{question}

\begin{question}
Higher dimensional parallelotopes (affine images of cubes) are IDP. Do they have unimodular covers? 
\end{question}



%\begin{proof}
%Let $P$ be a smooth centrally symmetric lattice $3$-polytope.
%Beck et al.~\cite{BHHHJKM2019} show that $P$ is IDP by covering it with parallelepipeds and unimodular simplices. Once we know parallelepipeds have unimodular covers (\Cref{thm:parallelepipeds}), we conclude $P$ has a unimodular cover too.
%\end{proof}

The two-dimensional case of \Cref{conj:Oda}\eqref{itm:mixedIDP} is known to hold, with three different proofs by Fakhruddin~\cite{Fakhruddin}, Ogata~\cite{Ogata} and Haase et al.~\cite{HNPS2008}. This last one actually shows that smoothness of $Q$ is not needed. In dimension three, however, the conjecture fails without the smoothness assumption. Indeed, it fails if $P=Q$ is any non-unimodular \emph{empty tetrahedron}, by which we mean a lattice tetrahedron containing no lattice points other than its vertices (see the proof of \Cref{lemma:corner} for a classification of empty tetrahedra).
\paco{added def of empty tetrahedron}\giulia{I didn't like "details on them" and changed it to  "classification of empty tetrahedra". Is this ok?}

An alternative approach to \Cref{conj:Oda}\eqref{itm:mixedIDP} is via Cayley sums, which we discuss in  \Cref{sec:cayley}. 
Recall that the \emph{Cayley sum} of two lattice polytopes $Q_1,Q_2\subset \R^d$ is the lattice polytope
\[
\cayley(Q_1,Q_2) := \conv(Q_1\times\{0\} \cup Q_2 \times \{1\}) \subset \R^3.
\]
We normally require $\cayley(P,Q)$ to be full-dimensional (otherwise we can delete coordinates) but for this does not need $Q_1$ and $Q_2$ to be full-dimensional. It only requires the linear subspaces parallel to them to span $\R^d$.

As we note in \Cref{prop:mixedIDP}, if the Cayley sum of $P$ and $Q$ is IDP then the pair $(P,Q)$ is IDP.
In particular, the following statement from \Cref{sec:cayley} is stronger than the afore-mentioned result of \cite{Fakhruddin,HNPS2008,Ogata}:

\begin{theorem}
\label{thm:cayley}
Let $P$ and $Q$ be lattice polygons, with $P$ a weak Minkowski summand of $Q$. Then the Cayley sum $\cayley(P,Q)$ has a unimodular cover.
\end{theorem}

This has the following two consequences, also proved in \Cref{sec:cayley}.
A  \emph{prismatoid} is a polytope whose vertices all lie in two parallel facets. 
A polytope has width $1$ if its vertices lie in two \emph{consecutive} parallel lattice hyperplanes. Observe that this is the same as being ($SL(\Z,d)$-equivalent to) a Cayley sum.

\begin{corollary}
\label{coro:prismatoid}
Every smooth $3$-dimensional lattice prismatoid has a unimodular cover.
\end{corollary}

\begin{corollary}
\label{coro:width1}
Every integer dilation $kP$, $k\ge 2$, of a lattice $3$-polytope $P$ of width $1$ has a unimodular cover.
\end{corollary}

A special case of the latter are integer dilations of empty tetrahedra. That their dilations have
unimodular covers is \cite[Cor.~4.2]{SantosZiegler} (and is also implicit in \cite{KantorSarkaria}).

\medskip
We believe that the $3$-polytopes in all these statements have unimodular triangulations, but this remains an open question.


\subsection*{Acknowledgements:} We thank Akiyoshi Tsuchiya, Spencer Backman, and Johannes Hofscheier for posing these questions to us and 
Christian Haase for helpful discussions.


\section{Parallelepipeds}
\label{sec:parallelepipeds}

The main tool for the proof of \Cref{thm:parallelepipeds} is what we call the parallelepiped circumscribed to a given tetrahedron, defined as follows:

\begin{definition}
\label{def:circunpara}
Let $T$ be a tetrahedron with vertices $p_1$, $p_2$, $p_3$, and $p_4$. Consider the points $q_i= \frac12 (p_1+p_2+p_3+p_4) - p_i$, $i\in [4]$, and let
\[
C(T)=\conv(p_i,q_i: i\in[4]).
\] 
$C(T)$ is a parallelepiped with facets $\conv(p_i, p_j, q_k, q_l)$ for all choices of $\{i,j,k,l\}=[4]$. We call it the \emph{parallelepiped circumscribed} to $T$.

For each $i \in [4]$, let $T_i=\conv(q_i, p_j, p_k, p_l)$, with $\{i,j,k,l\}=[4]$; we call these $T_i$ the \emph{corner tetrahedra} of $C(T)$. Together with $T$ they triangulate of $C(T)$.
\end{definition}

The idea to have in mind is that of the regular tetrahedron inscribed in a cube. 
Modulo an affine transformation this is exactly the situation of $T$ and $C(T)$.
\giulia{we should add a picture of the standard cube}
\paco{perhaps, but not necessarily}

%\Giulia{not sure if we need this remark:
%$S$ is ``half a fundamental domain'' for the affine lattice $\Lambda_P$ generated by $p_1,\dots,p_4$, in the following precise sense: if $x$ is an arbitrary point, then $S$ intersects either $\Lambda_P + x$ or $\Lambda_P -x$ (and, if $x$ is generic then the intersection is unique). Proof: think of this property for the standard cube and the lattice $\{sum of coordinates is even\}$ generated by the vertices of the regular tetrahedron inscribed in it. (Remark: let us assume without loss of generality that $\Lambda_P$ contains the origin, so that $\Lambda + x$ is the class of $x$ modulo $\Lambda$).
%}

\begin{lemma}
\label{lemma:corner}
Let $T=\conv\{p_1,p_2,p_3,p_4\}$ be an empty lattice tetrahedron that is not unimodular. Let $C(T)$ be the parallelepiped circumscribed to $T$ and let $T_1, T_2,T_3$ and $T_4$ be the corresponding corner tetrahedra in $C(T)$. Then, every $T_i$ contains at least one lattice point different from $\{p_1,\dots,p_4\}$.
\end{lemma}

\begin{proof}
By White's classification of empty tetrahedra (\cite{White1964}, see also, e.~g.~\cite[Sect.~4.1]{HPPS-survey}), there is no loss of generality in assuming $T=\conv(p_1,p_2,p_3,p_4)$ with
\[
p_1=(0,0,0), \quad
p_2=(1,0,0), \quad
p_3=(0,0,1), \quad
p_4=(a,b,1).
\]
where $b\ge 2$ is the (normalized) volume of $T$, and $a\in \{1,\dots,b-1\}$ satisfies $\gcd(a,b)=1$. This gives 
\paco{picture of the quadrilaterals $p_1q_4p_2q_3$ and $q_2p_3q_1p_4$ would be good.}
\paco{perhaps, but not necessarily}
\begin{align*}
q_1=\left(\frac{1+a}2,\frac{b}2,1\right), &&
q_2=\left(\frac{a-1}2,\frac{b}2,1\right), \quad\\
q_3=\left(\frac{1+a}2,\frac{b}2,0\right), &&
q_4=\left(\frac{1-a}2,-\frac{b}2,0\right).
\end{align*}

Then, the inequalities $b\ge 1+a \ge 2$ imply:
\[
u:=(1,1,0)\in \conv(p_1p_2q_3) \subset T_4, \quad
v:=(0,-1,0)\in \conv(p_1p_2q_4) \subset T_3.
\]

Observe that $u+v=p_1+p_2=q_3+q_4$.
Now, the translation of vector $\left(\frac{a-1}2,\frac{b}2,1\right)$ sends the facet $\conv(p_1p_2q_3q_4)$ of $C(T)$ to the opposite facet $\conv(q_2q_1p_4p_3)$, so that the translated points $u + \left(\frac{a-1}2,\frac{b}2,1\right)$ and $v + \left(\frac{a-1}2,\frac{b}2,1\right)$ lie in $\conv(q_2q_1p_4p_3)$ and are centrally symmetrically placed in it. Thus, one of them belongs to $\conv(q_1p_4p_3) \subset T_2$ and the other to $\conv(q_2p_4p_3) \subset T_1$.
%\paco{Alternative arguments: one can explicitly compute coordinates of lattice points in $T_1$ and $T_2$, or one can say that the isomorphism $T_{a,b} \cong T_{a^{-1},b}$ of White tetrahedra exchanges the edges at $\{z=0\}$ and at $\{z=1\}$.}
%Now, this implies that the quadrilateral $\conv(p_1q_4p_2q_3)$ contains a fundamental domain for the lattice $\Z^2\times\{0\}$. Hence, its translate $\conv(q_2p_3q_1p_4)$ contains a fundamental domain for $\Z^2\times\{1\}$ and, in particular, it contains at least one lattice point other than $p_3$ and $p_4$. By central symmetry around its center $\left(\frac{a}2,-\frac{b}2,1\right)$, $\conv(q_2p_3q_1p_4)$ must contain lattice points in both triangles $\conv(q_2p_3p_4)\subset T_1$ and $\conv(q_1p_3p_4)\subset T_2$.
\end{proof}


\begin{lemma}
\label{lemma:3<4}
Let $P$ be a lattice parallelepiped and let $T\subset P$ be a tetrahedron. Then, at least one of the four corner tetrahedra $T_i$ of the circumscribed parallelogram $C(T)$ is fully contained in $P$.
\end{lemma}

\begin{proof}
Let us denote the vertices of $T$ by $p_1, p_2, p_3, p_4$ and the vertices of $C(T)$ not in $T$ by $q_1, q_2, q_3, q_4$, with the conventions of \Cref{def:circunpara}. 

We call \emph{band} any region of the form $f^{-1}([\alpha,\beta])$ for some functional $f\in (\R^3)^*$ and closed interval $[\alpha,\beta]\subset \R$.
We claim that any band containing $T$ must contain at least three of the $q_i$s. 
This claim implies that the parallelepiped $P$, which is the intersection of three bands, contains at least one of the $q_i$s and hence it fully contains the corresponding $T_i$.

To prove the claim, suppose that $q_1\not\in B:= f^{-1}([\alpha,\beta])$ for a certain band $B \supset T$. 
Without loss of generality, say $f(q_1)<\alpha$. Then the equalities $q_1+q_2=p_3+p_4$ and $q_1+p_1=q_2+p_2$ respectively give:
\begin{gather}
\label{eq:first}
f(q_2) = f(p_3+p_4-q_1) = f(p_3)+f(p_4)-f(q_1) > 2\alpha-\alpha=\alpha,\\
\label{eq:second}
f(q_2) = f(q_2+p_2-p_1) = f(q_2)+(f(p_2)-f(p_1)) < \alpha + (\beta-\alpha) = \beta,
\end{gather}
so that $q_2 \in B$.

Inequality \eqref{eq:first} also implies
\begin{equation}
\label{eq:third}
f(q_1) <  f(p_i) < f(q_2), \quad\text{ for $i=3,4$}.
\end{equation}

The translation of vector $\frac12 (p_1+p_2-p_3-p_4)$ sends $q_1,q_2,p_3,p_4$ to $p_2,p_1,q_4,q_3$ (in this order). By applying this to inequality \eqref{eq:third}, we obtain
\[
\alpha \le f(p_2) < f(q_i) < f(p_1) \le \beta, \quad\text{ for $i=3,4$},
\]
so that $q_3,q_4 \in B$.
%
This finishes the proof of the claim, and of the lemma.
%
%Let $f\in (\R^3)^*$ be any functional. Fix $i \in [4]$; observe that if $f(q_i) > f( p_j)$ for all $j\neq i$, then 
%\begin{gather*}
%f(q_i) \overset{(i)}{>} f(q_j) \overset{(ii)}{>} f(p_i) \\
%f(q_i) \overset{(iii)}{>} f(p_j) \overset{(iv)}{>} f(p_i) \\
%f(p_j)\overset{(v)}{>}f(q_k)
%\end{gather*}
%for all $j \neq i$, $k \notin \{j, i\}$. 
%
%Indeed, for any $\{i,j,k,l\}=[4]$ we have 
%
%\begin{align*}
%f(q_i) > f( p_j) \Leftrightarrow \frac12 f(p_l+&p_k -p_j-p_i) > 0  \Leftrightarrow f(q_j) > f(p_i)\\
%& \Updownarrow \\
% f(q_l) < f(p_k) \Leftrightarrow \frac12 f(-p_l-&p_k +p_j+p_i) < 0 \Leftrightarrow f(q_k) < f(p_l),
%\end{align*}
%which shows the equivalence of $(i)$, $(iii)$ and $(v)$; note that $(iii)$ is our assumption. Inequalities $(ii)$ and $(iv)$ follow by taking the sum of two inequalities of the form $\frac12 f(p_l+p_k -p_j-p_i) > 0$. 
%
%Using this observation, we can show that each of pairs of parallel facets of $P$ cuts out at most one of the $q_i$s:
%if $q_i$ lies outside of $P$, it must violate a facet-defining inequality $f(x) \leq a$.Thus $f(q_i) > a \geq f(p_j)$, and by our observation, it follows that all remaining $q_j$ satisfy the inequality $f(x) \leq a$. Since $P$ is a parallelogram, there is a facet with inequality $f(x) \geq b$; by what is said above, the minimum value of $f(x)$ is attained at $p_i$, which is in $P$. Thus $f(q_j) > f(p_i) \geq b$, that is, the inequality is satisfied by all the $q_j$.
%
%Since at most one of the $q_i$ does not satisfy the inequalities corresponding to a pairs of parallel facets of $P$, and there are only three pairs of facets, one of $q_i$s must satisfy all the inequalities and therefore be inside of $P$.
\end{proof}
%\giulia{here it may also be nice to add a picture}

\begin{corollary}
\label{coro:coverpara}
Let $T$ be an empty lattice tetrahedron contained in a lattice parallelepiped $P$. Then, $T$ can be covered by unimodular tetrahedra contained in $P$.
\end{corollary}

\begin{proof}
We proceed by induction on the (normalized) volume of $T$, which is a positive integer. If this volume equals $1$ then $T$ is unimodular and there is nothing to prove, so we assume $T$ is not unimodular. Let $p_1, p_2, p_3, p_4$ denote the vertices of $T$.

\Cref{lemma:3<4} guarantees that one of the corner tetrahedra $T_i$ of the parallelepiped $C(T)$ is contained in $P$. Without loss of generality, suppose $T_4 = \conv(p_1, p_2, p_3,q_4)$ is in $P$. By \Cref{lemma:corner}, we know that $T_4$ contains a lattice point other than the $p_i$s, which we denote by $u$. 
%Observe that from the proof of \Cref{lemma:corner} we know more: $u$ lies in the face $\conv(p_1, p_2, q_4)$.\giulia{we need this for the volume computation below. perhaps it should be made explicit in the statemente of the lemma...}
%
Then $S=\conv(T\cup \{u\})$ can be triangulated in two different ways: $S=T \cup T'_4$, where $T'_4 = \conv(p_1, p_2, p_3, u) \subseteq T_4$ and $S= S_1 \cup S_2 \cup S_3$, with
\[
S_1= \conv(p_2,p_3,p_4, u),
S_2=\conv(p_1,p_3,p_4, u),
S_3=\conv(p_1,p_2,p_4, u).
\]

Each of the tetrahedra $S_i$ has lattice volume strictly smaller than $T$ because, for each $i$, $p_i$ is the unique point of $C(T)$ maximizing the distance to the opposite facet $\conv(p_j,p_k,p_l)$ of $T$. Thus, $S_1$, $S_2$ and $S_3$ cover $T$ and have volume strictly smaller than $T$. The $S_i$ may not be empty, but we can triangulate them into empty tetrahedra, which by inductive hypothesis they can be covered unimodularly.
%
%Observe that  $T'_4$ and/or one of the $S_i$ can degenerate to a triangle, if $u$ lies in $\conv(p_1,p_2,p_3)$, but this does not affect the argument.
%
%: indeed, $\Vol(S_3)=\Vol(T'_4)$, since these tetrahedra have a common face $\conv(p_1, p_2, l)$ and the remaining vertex is at distance one from this face in both. Now $T'_4$ is contained in $T_4$, and we know that each corner tetrahedron $T_i$ has volume half that of $T$. The remaining two tetrahedra $S_1$ and $S_2$ must therefore have volume summing to that of $T$, and therefore each has volume strictly less than $T$.
\end{proof}

\begin{proof}[Proof of \Cref{thm:parallelepipeds}]
Arbitrarily triangulate the parallelepiped into empty lattice tetrahedra and apply \Cref{coro:coverpara} to these tetrahedra.
%Let $P \subseteq \R^3$ be a lattice parallelepiped. It can be triangulated into empty lattice tetrahedra, simply by taking a triangulation which uses all lattice points as vertices.  If all the tetrahedra in the triangulation are unimodular we are done. Otherwise, suppose $T$ is an empty tetrahedron in the triangulation which is not unimodular. \Cref{lemma:3<4} guarantees that one of the corner tetrahedra $T_i$ of the parallelepiped $C(T)$ circumscribed to $T$ is contained in $P$. Without loss of generality, suppose $T_4 = \conv(p_1, p_2, p_3,q_4)$ is in $P$. By \Cref{lemma:corner}, we know that $T_4$ contains a lattice point, which we denote by $u$. 
%
%
%This allows us to cover $T$ with lattice simplices of smaller volume recursively, until it is covered by unimodular ones.
\end{proof}



%\begin{remark}
%\label{rem:octahedra_and_tri-prisms}

%We finish with a question which, if answered in the positive, would imply \Cref{conj:Oda}\eqref{itm:smoothIDP} in dimension three.
Let us say that a lattice $3$-polytope $P$ \emph{has the circumscribed parallelepiped property} if it satisfies the conclusion of \Cref{lemma:3<4}:  ``for every empty tetrahedron $T$ contained in $P$ at least one of the four corner tetrahedra in $C(T)$ is 
contained in $P$''. 
If this holds then $P$ has a unimodular cover, since then the proofs of \Cref{coro:coverpara} and \Cref{thm:parallelepipeds} work for $P$.
Hence, a positive answer to the following question would imply that every smooth $3$-polytope has a unimodular cover, which in turn implies \Cref{conj:Oda}\eqref{itm:smoothIDP} in dimension three. 


\begin{question}
Does every smooth 3-polytope have the circumscribed parallelepiped property? 
\end{question}



Our proof that parallelepipeds have the property (\Cref{lemma:3<4}) is based on the fact that they have only three (pairs of) normal vectors. The proof, and the property of being IDP, fail if there are four of them:

\begin{example}[Non-IDP octahedron and triangular prism]
\label{ex:non-IDP}
The following lattice octahedron $Q$ and triangular prism $P$ are not IDP:
\begin{gather}
Q= \conv((0,1,1),(1,0,1),(1,1,0),(0,-1,-1),(-1,0,-1),(-1,-1,0)),\\
P=\conv((0,1,1),(1,0,1),(1,1,0),(-1,0,0),(0,-1,0),(0,0,-1)).
\end{gather}

Indeed, in both polytopes the only lattice points are the six vertices and the origin. The point $(1,1,1)$ lies in the second dilation but is not the sum of two lattice points in the polytope. Hence, they are not IDP, which implies they do not admit unimodular covers.
\end{example}

\section{ Cayley sums}
\label{sec:cayley}
Let $P$ and $Q$ be two lattice polytopes in $\R^d$. We do not require them to be full-dimensional, but we assume their Minkowski sum is. Remember that the \emph{Minkowski sum} $P+Q$ and the \emph{Cayley sum} of $P$ and $Q$ are defined as:
\begin{gather*}
P + Q := \{ p+q \in \R^d: p\in P, q \in Q\}.\\
\cayley (P,Q) = \conv( P\times\{0\} \cup Q\times \{1\}) \in \R^{d+1}.
\end{gather*}

\paco{removed definitions of "nef cayley", "mixed IDP" and "weak Minkowski sum" from here}
%\begin{definition}
%In the above conditions, we say that $Q$ \emph{is a weak Minkowski summand of $P$} if the normal fan of $Q$ refines that of $P$. 
%If this happens, we say that $\cayley(P,Q)$ is a \emph{nef Cayley sum}.\giulia{I guess there's no nice term for this now...}
%\paco{not sure; maybe we can keep nef after all; another (unclear) option could be ``weak prism over $Q$'' or ``degenerate prism over $Q$''.}
%
%We say that the pair $(P,Q)$ is \emph{mixed IDP} if 
%\paco{``mixed IDP'' is not the right property, since this is weaker than IDP} \giulia{do we give it a name at all then?}
%\paco{perhaps mixed-IDP}
%\[
%(P\cap \Z^d) + (Q\cap \Z^d) = (P+Q)\cap \Z^d.
%\]
%\end{definition}
%
%In this language, \Cref{conj:Oda}\eqref{itm:mixedIDP} translates to:
%\emph{If $Q$ is nef for $P$ and $P$ is smooth then the pair $(P,Q)$ mixed IDP.}
%
The so-called \emph{Cayley Trick} is the isomorphism
\[
2\cayley(P, Q) \cap (\R^d\times \{1\}) \cong P+Q,
\]
which easily implies:

\begin{proposition}[see, e.g.~\protect{\cite[Thm.~0.4]{Tsuchiya}}]
\label{prop:mixedIDP}
If $\cayley(P,Q)$ is IDP then the pair $(P,Q)$ is mixed IDP.
\end{proposition}

%The assumption that $P+Q$ is $d$-dimensional is equivalent to $\cayley (P,Q)$ being $(d+1)$-dimensional.

The Cayley Trick also provides the following canonical bijections:

\[
\begin{array}{ccc}
\text{polyhedral subdivisions of $\cayley(P,Q)$} &\leftrightarrow& \text{mixed subdivisions of $P + Q$}\\
\text{triangulations of $\cayley(P,Q)$} &\leftrightarrow& \text{fine mixed subdivisions of $P + Q$}\\
\text{unimodular simplices in $\cayley(P,Q)$} &\leftrightarrow& \text{unimodular prod-simplices in $P + Q$}.
\end{array}
\]
See \cite{DLRS2010} for more details on the Cayley Trick and on triangulations and polyhedral subdivisions of polytopes.
In fact these bijections can be taken as definitions of the objects in the right-hand sides. In particular, 
we call \emph{prod-simplices} in $P+Q$ the Minkowski sums $T_1+T_2$ where $T_1\subset P$ and $T_2\subset Q$ are simplices with complementary affine spans. A prod-simplex is \emph{unimodular} if the edge vectors from a vertex of $T_1$ and from a vertex of $T_2$ form a unimodular basis.

\medskip

We now turn our attention to $d=2$, in order to prove \Cref{thm:cayley}. A triangulation of $\cayley(P,Q)\subset \R^3$ consists of tetrahedra of types $(1,3)$, $(2,2)$ and $(3,1)$, where the type denotes how many vertices they have in $P$ and in $Q$. Empty tetrahedra of types $(1,3)$ or $(3,1)$, which are Cayley sums of a triangle in $P$ and a point in $Q$, or viceversa, are automatically unimodular. The case that we need to study are therefore tetrahedra of type $(2,2)$, which are Cayley sums of a segment $p\subset P$ and a segment $q\subset Q$.
%
%
%The unimodular prod-simplices are unimodular parallelograms $p+q$ with $p$ a segment of $P$ and $q$ a segment of $Q$, and unimodular triangles $T+u$, where $T \subset P$ and $u \in Q$ or viceversa. 
%
%Consider any lattice triangulation $\T$ of $\cayley(P,Q)$, or equivalently a mixed subdivision of $P+Q$. Observe that any simplex in $T$ has vertices in $P \times \{0\}$ or in $Q \times \{1\}$, since $\cayley(P,Q)$ lies between two consecutive lattice hyperplanes $\{x_3=0\}$ and $\{x_3=1\}$. Thus simplices in $\T$ are of three types; they can , or have three vertices in $P$ and one in $Q$ (type $(3,1)$), have one vertex in $P$ and three in $Q$ (type $(1,3)$) or have two vertices in $P$ and two in $Q$ (type $(2,2)$). 
%
\giulia{fix this}
\paco{done, but check it}
%
%As a consequence, we also have that:
%\begin{align*}
%\text{$\cayley(P,Q)$ has a unimodular cover} &\leftrightarrow \text{$P+Q$ can be covered by unimodular prod-simplices}.
%\end{align*}
The following lemma, whose proof we postpone to \Cref{sec:the_lemma}, is crucial to understand how to unimodularly cover these tetrahedra. 
%For this the following result, the proof of which we postpone to \Cref{sec:the_lemma}, is crucial. 
We use the following conventions: if $a, b$ are points, we denote by $[a,b]$ and $(a,b)$ respectively the closed and open segments with endpoints $a,b$. Given a segment $s=[a,b]$, we denote the vector $\vec s:= b-a$ and  the line spanned by $\vec s$ by $\vecline s$.

\begin{lemma}
\label{lemma:cayley}
%\paco{$Q$ is 2-dim, but $P$ may not}
Let $Q$ be a two-dimensional lattice polytope and $P$ a weak Minkowski summand of it. 
Let $p=[p_1,p_2] \subset P$ and $q=[q_1,q_2]\subset Q$ be two primitive and non-parallel lattice segments, and let $\vecline p$ and $ \vecline q$ be the lines spanned by them.  If the parallelogram $p + q$ is not unimodular, then at least one of the regions
\[
((p_1, p_2) + \vecline q ) \cap P, 
\qquad \text{and} \qquad
((q_1, q_2) + \vecline p ) \cap Q
\]
contains a lattice point. See \Cref{fig:strips}.
%\[
%(\int(p) + \vecline q) \cap P
%\qquad \text{or} \qquad
%\int(q + \vecline p) \cap Q
%\]
%contains a lattice point.
\end{lemma}
%\giulia{what I want is: a lattice point $p \in P$ such that either $p+q_1$ or $p+q_2$ lies in the interior of $\{p_1, p_2\} + \{q_1, q_2\}$ (or symmetric for $Q$)}

\begin{figure}[htb]
\scalebox{.75}{\input{strips.pdf_t}}
\caption{The strips  of Lemma \ref{lemma:cayley}}
\label{fig:strips}
\end{figure}

\begin{corollary}
\label{coro:covercayley}
\paco{inserted this corollary to make proof clearer, and to parallel section 2} \giulia{picture?}
Let $T$ be an empty lattice tetrahedron contained in the Cayley sum $\cayley(P,Q)$, where $P$ is a weak Minkowski summand of $Q$. Then, $T$ can be covered by unimodular tetrahedra contained in $\cayley(P,Q)$.
\end{corollary}

\begin{proof}
The proof is by induction on the normalized volume of $T$, which we assume to be at least $2$. This implies that $T$ is of type $(2,2)$,
since empty tetrahedra of types $(1,3)$ and $(3,1)$ are unimodular. Thus, $T$ is the Cayley sum of primitive segments $p=[p_1,p_2]\subset P$ and $q=[q_1,q_2]\subset Q$.  
Let $u$ be the lattice point whose existence is guaranteed by \Cref{lemma:cayley}. Assume  (the other case is similar) that 
\[
u \in ((p_1, p_2) + \vecline q ) \cap P,
\]
and call $t$ the triangle $t=\conv( u,  p_1,  p_2)\subset P$.

Let us denote $\tilde u$, $\tilde p_1$, $\tilde p_2$, $\tilde q_1$, $\tilde q_2$ the points corresponding to $u, p_1, p_2, q_1, q_2$ in $\cayley(P,Q)$.
That is, $\tilde p_i = p\times\{0\}$, $\tilde q_i = p\times\{1\}$, and $\tilde u = u\times\{0\}$.
Observe that the assumption $u\in((p_1, p_2) + \vecline q$ implies that of the segments $[u,q_i]$ crosses the triangle $\conv(p_1,p_2,q_j)$, where $\{i,j\}=\{1,2\}$. In turn, this means that the polytope $\conv(\tilde u, \tilde p_1, \tilde p_2, \tilde q_1, \tilde q_2) = \cayley(t,q)$
 has the following two triangulations:
\begin{gather*}
\mathcal T^+:= \left\{ \cayley(p,q), \cayley(t, \{q_i\}) \right\},
\\
\mathcal T^-:= \{ \cayley([p_1,u],q), \cayley([p_2,u],q), \cayley(t, \{q_j\}) \}.
\end{gather*}
The tetrahedra $\cayley(t, \{q_j\})$ and $\cayley(t, \{q_i\})$ are unimodular, which implies that $T=\cayley(p,q)$ has volume equal to the sum of the volumes of $\cayley([p_1,u],q)$ and $\cayley([p_2,u],q)$. In particular, we have covered $T$ by the three tetrahedra in $\mathcal T^-$, which are of smaller volume and hence have unimodular covers by inductive assumption.
%
%A unimodular triangulation of a simplex of type $(3,1)$ (resp $(1,3)$) is obtained simply by unimodularly triangulating the triangular face lying in $P$ (resp $Q$) and coning over the vertex in $Q$ (resp $P$). This leaves simplices of type $(2,2)$. In the mixed subdivision of $P+Q$ corresponding to $T$, these are mixed cells, that is parallelograms with one edge in $P$ and one in $Q$ \giulia{how imprecise can be without being confusing?}.
%
%Let $S=\{p_1, p_2\} + \{q_1, q_2\}$ be such a parallelogram. By \Cref{lemma:cayley} we know that there is a lattice point $p \in P$ such that either $p+q_1$ or $p+q_2$ lies in $S$, w.l.o.g. suppose it is $p+q_1$. Then we can cover $S$ by the triangle $\{p_1, p_2, p\} + q_1$ and parallelograms $\{p_1, p\}  + \{q_1, q_2\}$ and $\{p_2, p\} + \{q_1, q_2\}$. As above, the triangle can be unimodularly triangulated, while the volume of the two parallelograms sums up to the volume of $S$, and therefore each has volume strictly smaller than $S$. By iterating this procedure, we can cover $S$ by unimodular cells.
\end{proof}

\begin{proof}[Proof of \Cref{thm:cayley}]
Arbitrarily triangulate $\cayley(P,Q)$ into empty lattice tetrahedra and apply \Cref{coro:covercayley} to these tetrahedra.
\end{proof}

Let us now show how to derive \Cref{coro:prismatoid,coro:width1} from this theorem.
\emph{Prismatoids} were defined in~\cite{Santos-hirsch} as polytopes whose vertices all lie in two parallel facets. In particular, a \emph{lattice prismatoid} is any $d$-polytope $SL(\Z,d)$-equivalent to one of the form
\[
\conv(Q_1\times\{0\} \cup Q_2 \times \{k\}),
\]
where $Q_1,Q_2$ are lattice $(d-1)$-polytopes and $k\in \Z_{>0}$. This is almost a generalization of Cayley sums, which would be the case $k=1$, except the definition of prismatoid requires $Q_1$ and $Q_2$ to be full-dimensional, while the Cayley sum only requires this for $Q_1+Q_2$. \giulia{I didn't like "this is a generalization except...}

\begin{proposition}
\label{prop:prismatoid}
Let $Q_1$, $Q_2$ be two lattice polygons and consider the prismatoid 
\[
P:= \conv(Q_1\times\{0\} \cup Q_2 \times \{k\},
\]
with $k\ge 2$. 
%Assume that $P\cap(\R^2\times\{i\})$ is a lattice polytope for every $i=0,\dots, k$. This is equivalent to assuming that $P\cap(\R^2\times\{1\})$ is a lattice polytope:
If $P\cap(\R^2\times\{1\})$ is a lattice polygon then $P$ has a unimodular cover.
\end{proposition}

\begin{proof}
The condition that $P\cap(\R^2\times\{1\})$ is a lattice polygon implies the same for $P\cap(\R^2\times\{i\})$, for every $i$. 
Indeed, the condition implies that every edge of $\cayley(P,Q)$ of the form $[u\times \{0\}, v\times \{k\}]$ has a lattice point in $\R^2\times\{i\}$, and hence it has a lattice point in $P\cap(\R^2\times\{i\})$, for every $i$.

Observe that for every $i\in \{1,\dots,k-1\}$ the intersection $P\cap(\R^2\times\{i\})$ has the same normal fan as $Q_1+Q_2$. Thus, each slice
\[
P \cap (\R^2\times[i-1,i])
\]
is a Cayley polytope. If $i\in\{2,\dots,k-1\}$, both bases have the same normal fan (and therefore each is a weak Minkowski summand of the other), while for $i\in \{1,k\}$ one base is a weak Minkowski summand of the other. We can therefore apply \Cref{thm:cayley} to each slice and combine the covers thus obtained to get a unimodular cover of $P$.
\end{proof}


\begin{proof}[Proof of \Cref{coro:prismatoid,coro:width1}]
In both cases the polytope under study satisfies the hypotheses of \Cref{prop:prismatoid}: in \Cref{coro:prismatoid}, the smoothness of the prismatoid implies that every edge of the form $[u\times \{0\}, v\times \{k\}]$ has lattice points in all slices. In \Cref{coro:width1}, since $P$ has width one, $P\cong \cayley(Q_1, Q_2)$ for some $Q_1$ and $Q_2$. Hence,
\[
kP \cap(\R^2\times\{1\}) = (k-1)Q_1 + Q_2.
\qedhere
\] 
\end{proof}




\section{Proof of \Cref{lemma:cayley}}
\label{sec:the_lemma}
Let $f_q$ be the primitive lattice functional constant on $q$ and $f_p$ the one constant on $p$. We assume that $f_q(p_1) < f_q(p_2)$ and $f_p(q_1) < f_p(q_2)$.

Observe that in the strip $q +\vecline p$, there is a unique lattice point on the line $f_q(x)=1$;  indeed, since $q$ is primitive, the only way that in the strip there could be two lattice points on $f_q(x)=1$ is if they were on the boundary of the strip, which would however imply that $p+q$ is a unimodular paralellogram, against our assumptions.
%We call that point $q^+$. Similarly, $q +\vecline p$ has a unique lattice point at $f_q(x)=-1$, which we call $q_-$. Observe that $q_++q_-=q_1+q_2$ and that the triangles $\conv(q_1, q_2, q_+)$ and $\conv(q_1, q_2, q_1)$ are unimodular.
Since translating the polytopes by lattice vectors will not result in any loss of generality, we can assume that $p_1$ is that unique lattice point. That is, $f_q(p_1)=-1$, or equivalently, the triangle $\conv(q_1, q_2, p_1)$ is unimodular. Similarly, the unique lattice point in the strip on the line $f_q(x)=1$ is then $q_1+q_2 -p_1$.

We let $H_1=\{f_q(x) < 0\}$ and $H_2=\{f_q(x) > 0\}$; similarly let $V_1=\{f_p(x) < 0\}$ and $V_2=\{f_p(x) > 0\}$. 
In the figures, we draw $p$ as a vertical segment and $q$ as a horizontal one, so that $H_i \cap V_j$ are the four quadrants. 

Let $w=\area(p+q) \geq 2$, where $\area$ denotes the area normalized to a fundamental domain. Then 
\[
w=\width_{f_q}(p + \vecline q )=\width_{f_q}(p)=\width_{f_p}(q)=\width_{f_p}(q +\vecline p ). 
\] 
\paco{picture for setup needs change}

\begin{figure}
\includegraphics[scale=.3]{setup.png}
\caption{Setup for \Cref{lemma:cayley}}
\end{figure}

\begin{proof}[Proof of \Cref{lemma:cayley}]
Suppose by contradiction that there is no lattice point as described in the lemma. In particular, no vertex of $P$ can be in the strip $p + \vecline q$.  Thus $P$ has two primitive segments in the boundary which each have one vertex on each side of the strip $p + \vecline q$; we will call these $\ell=[l_1, l_2]$ and $r=[r_1, r_2]$, with $\ell$ and $r$ crossing the strip $p + \vecline q$  in $V_1$ and $ V_2$ respectively and the convention that $f_q(l_2) >f_q(l_1)$ and $f_q(r_2) >f_q(r_1)$. This readily implies 
\begin{gather}
\label{eq:widthp}
\begin{array}{cc}
f_q(l_1) \leq f_q(p_1), &
f_q(l_2) \geq f_q(p_2), \\
f_q(r_1) \leq f_q(p_1),&
f_q(r_2) \geq f_q(p_2).
\end{array}
\end{gather}


The same holds for $Q$ and the strip $q+\vecline p$, and we call the segments  $b=[b_1, b_2], t=[t_1, t_2]$ with $b$ and $t$ crossing the strip in $H_1$ and $H_2$ respectively. Again we have
\begin{gather}
\label{eq:widthq}
\begin{array}{cc}
f_p(t_1) \leq f_p(q_1), &
f_p(t_2) \geq f_p(q_2), \\
f_p(b_1) \leq f_p(q_1), &
f_p(b_2) \geq f_p(q_2).
\end{array}
\end{gather}

Observe that a priori one of $l$ and $r$ can coincide with $p$, if this is on the boundary of $P$, and similarly one of $t,b$ might be $q$, if this is on the boundary of $Q$. 
\giulia{p=P ???}

%To make things more explicit, we assume the following coordinates for the edge-vectors:
%\[
%\vec l = (l_1,l_2), \quad
%\vec r= (r_1,r_2), \quad
%\vec b= (b_1,b_2), \quad
%\vec t= (t_1,t_2).
%\]
%Without loss of generality we assume $l_2,r_2,b_1,t_1 >0$. (They are non-zero and we can arbitrarily choose a sign for each vector).

%\begin{claim}
%$l_2, r_2, b_1, t_1 >1$.
%\end{claim}

\begin{claim}
The following inequalities hold, 
\begin{align*}
\width_{f_q}(\ell) , 
\width_{f_q}(r) ,
\width_{f_p}(t) ,
\width_{f_p}(b) \geq w.
\end{align*}
Each inequality is strict, unless the segment in question coincides with $p$ or $q$.
\end{claim}

\begin{proof}
The inequality $\geq w$ follows in each case from \eqref{eq:widthp} and \eqref{eq:widthq}.

If one of the inequalities, say the one for $\ell$, is not strict, then $\ell$ has one endpoint on each of the boundary lines of $(p + \vecline q)$. Unless $\ell = p$, one of the endpoints of $\ell$ is not an endpoint of $p$, say $l_1 \neq p_1$. Thus the triangle $T=\conv(p_2, p_1, l_1)$ is contained in $P$ and its edge $[p_1, l_1]$ is an integer dilation of $q$. Since $\width_{f_q}(T) =w \geq 2$, $T$ must contain a lattice point in the interior of the strip.
\end{proof}

%\begin{claim}
%$b_2$ and $t_2$ are non-zero and have the same sign. W.l.o.g.~we assume they are positive.
%\end{claim}
%
%\begin{proof}
%If, say, $b_2 \le 0 \le t_2$. Then the intersections of $t$ and $b$ with the vertical line $\{x=-\beta\}$ through $q_1$ are less than distance one apart (because $t$ must go below every lattice point in the interior of $p_1p_2 + q_1q_2 - p_1$ and $b$ above every latttice point in the interior of $p_1p_2 + q_1q_2 - p_2$). This implies the width w.r.t to $f_q$, the functional constant on $q_1q_2$, of any segment with vertices in $Q \cap \{x\le  -\beta\}$ is less than one, so it is impossible to fit a translated copy of $l$ in it.
%\end{proof}



\begin{claim}
\label{claim:b_and_t}
$f_q(b_2-b_1)$ and $f_q(t_2 - t_1)$ are non-zero and have the same sign. That is, $f_q$ achieves its maximum over $b$ and over $t$ on the same halfplane $V_1$ or $V_2$.
\end{claim}



%\begin{claim}
%The slopes of $b$ and $t$ must either be both strictly larger than the slope of $q$ or both be strictly smaller. W.l.o.g.~we assume they are larger.
%\end{claim}
%
%\begin{proof}
%If, say, $b$ has positive slope and $t$ negative. Then since $t$ intersects the vertical line $x=0$ spanned by $p$ below $p_2$, and $b$ intersects it above $p_1$, the slopes of $b$ and $t$ imply that  

\begin{proof}
Suppose by contradiction that the maximum of $f_q$ on $t$  lies in $V_1$ and that the maximum on $b$ lies in $V_2$. 

Then $Q \cap V_2$ is contained in the open strip $\{-1<f_q(x)<w-1\}$, of width $w$. This cannot contain a translated copy of $r$, since $\width_{f_q}(r) \geq w$, see \Cref{fig:claim2} .
\end{proof}

\begin{figure}[htb]
\scalebox{.75}{\input{claim2.pdf_t}}
\caption{Illustration of the proof of \Cref{claim:b_and_t}}
\label{fig:claim2}
\end{figure}

%\Giulia{fix it to look more unimodular at the bottom}
%\begin{figure}[htb]
%  \centering
%\begin{tikzpicture}[mynode/.style={circle, inner sep=1.5pt, fill=red}, scale =
%  1]
%%coordinate axes
%   \tkzInit[xmax=3.5,ymax=3.5,xmin=-3.5,ymin=-3.5]
%%        \tkzLabelX[orig=false,label options={font=\tiny}]
%%        \tkzLabelY[orig=false,label options={font=\tiny}]
%%        \tkzGrid
%%        \tkzDrawX
%%        \tkzDrawY
%
%%edges b, t
%  \node[
%%label={[shift={(0,-0.67)}]$b_1$}, 
%mynode] (b1) at (-3,-2) {}; 
%  \node[
%%label={[shift={(0,-0.67)}]$b_2$}, 
%mynode] (b2) at (2, 0.2) {};  
%  \node[
%%label={[shift={(0,0)}]$t_1$}, 
%mynode] (t1) at (-2.7, 1) {};  
%  \node[
%%label={[shift={(0,0)}]$t_2$}, 
%mynode] (t2) at (1.6,0.7) {};  
%
%%shaded region where r can be  
%  \draw[fill=gray!80, nearly transparent] (2.93,0.61) -- (0,0.81)  -- (0,-0.68) -- cycle;
%
%
%%p
%  \node[
%%label={[shift={(0,-0.67)}]$p_1$}, 
%mynode] (p1) at (0,-1) {}; 
%  \node[
%%label={[shift={(0,0)}]$p_2$}, 
%mynode] (p2) at (0,1) {};  
%%  \node[{p}]  at (-0.3,0.95)
%
%
%%q
%  \node[
%%label={[shift={(0,-0.67)}]$q_1$}, 
%mynode] (q1) at (-1.7,-0.4) {}; 
%  \node[
%%label={[shift={(0,-0.67)}]$q_2$}, 
%mynode] (q2) at (0.3,-0.4) {};  
%
%%segments b, t, p, q
%  \draw[ name path = b, black, thick] (b1) node[below, xshift=1.7cm, yshift=0.7cm] {b} -- (b2); 
%  \draw[name path = t, black, thick] (t1) node[below, xshift=1.2cm, yshift=-0.1cm] {t} -- (t2);
%  \draw[name path = p, black, thick] (p1) node[left, xshift=0 cm, yshift=1.3cm] {p} -- (p2);
%  \draw[name path = q, black, thick] (q1) node[below, xshift=0.4cm, yshift=0cm] {q} -- (q2);
%
%
%%prolonged lines through b and t
% \tkzDrawLine[add=0.2 and 0.3, color=black, dashed](b1, b2);
% \tkzDrawLine[ add=0.3 and 0.4, color=black, dashed](t1, t2);
% 
%%functional f_q
%  \tkzDefLine[orthogonal=through p1](p1,p2)
%  \tkzDrawLine[add = 2 and 1,color=black, dashed](p1,tkzPointResult)
%  \node[label={\small $\{f_q(x)=-1\}$}, xshift=0.4cm, yshift=-0.2cm] (lab1) at (3.3, -1) {};
%
%
%  \tkzDefLine[orthogonal=through p2](p1,p2) 
%  \tkzDrawLine[add = 2 and 1,color=black, dashed](p2,tkzPointResult) 
%    \node[label={\small $\{f_q(x)=w-1\}$}, xshift=0.4cm, yshift=-0.2cm] (lab2) at (3.3, 1) {};
%
%\end{tikzpicture}
%\caption{Picture for Claim 4.2}
%\label{fig:claim2}
%\end{figure}





We assume w.l.o.g. that the maximum on $t$ (and hence on $b$) is achieved in $V_2$ \giulia{do we need it??}, that is to say, $f_p$ and $f_q$ increase in the same direction along $t$ (and hence along $b$). 

\begin{claim}
\label{claim:r}
Assume w.l.o.g.~that $b$ and $t$ either are parallel or their affine spans cross in $V_2$. Then, 
\begin{enumerate}
\item The intersection of $Q$ with any line parallel to $p$ in $V_2$ has width w.r.t. $f_q$ strictly smaller than $w$.
\item $f_p(r_2) > f_p(r_1)$, that is, $f_p$ achieves its maximum over $r$ in $H_2$.
\end{enumerate}
\end{claim}

\begin{proof}
Both $t$ and $b$ must intersect $p$, otherwise $p_1$ or $p_2$ are the lattice points we are looking for in $Q$.  Their intersections with $p$ are thus endpoints of a segment of width w.r.t $f_q$ less than $w$, the width of $p$. Since $t$ and $b$ cross in $V_2$, the same is true for any segment parallel to $p$ contained in $Q \cap V_2$.  

%\begin{figure}
%\includegraphics[ width=.7\textwidth]{Claim43.jpg}
%\caption{Figure for Claim 4.3}
%\label{fig:claim3}
%\end{figure}
\begin{figure}[htb]
\scalebox{.75}{\input{claim3.pdf_t}}
\caption{Illustration of the proof of \Cref{claim:r}}
\label{fig:claim3}
\end{figure}

For part (b), If $f_p(r_2) \leq f_p(r_1)$, it would be impossible to fit a translated copy $r'$ of $r$ in the correct side of $Q$: $r$ would need to lie inside the triangle delimited by the affine line $\langle t \rangle$ and the inequalities $f_q(x) \geq f_q(r_1)$, $f_p(x) \leq f_p(r_1)$. However, this triangular region has width less than $w$  w.r.t. $f_q$, by combining part (a) with the fact that $f_p$ and $f_q$ increase in the same direction along $t$, see \Cref{fig:claim3}. 
\end{proof}

The last two claims can be summarized as saying that in the pictures $b$, $t$ and $r$ have positive slope. Observe that this implies that $q$ is not in the boundary of $Q$ and $p \neq r$, so both $P$ and $Q$ are full dimensional.

%Starting here we assume index 2.
%
%\begin{claim}
%The slopes of $t$ and $b$ are strictly smaller than one and that of $r$ strictly greater than one.
%\end{claim}
%
%\begin{proof}
%Otherwise the points $(0,1/2)$, $(0,-1/2)$, $(1/2,0)$, respectively, would be in the corresponding band.
%\end{proof}
%
%We are now ready to show a contradiction. Let $u$ and $v$ be the end-points of the translated copy of $r$ in $Q$, with $\vec r=v-u$. Let $d$ be the line of slope $1$ through $u$, which is a lattice line. Let $d'$ be the next parallel lattice line above $d$. By Claim A.3.1, the intersection of $d'$ with the vertical line through $u$ is above $Q$. Since the slope of $t$ is less than one, the same happens for every point of $d'$ to the right of $u$. In particular, $v$ must be below $d'$ and hence it must be on or below $d$. This contradicts the fact that the slope of $r$ is strictly greater than one. QED

Let $g$ be the primitive lattice functional constant on $[p_1, q_2]$ (and therefore constant also on $[q_1, q_1+q_2-p_1]$). By the assumption on $p_1$, the values of $g$ on these segments differer by $1$. We choose the sign of $g$ so that 
\[
g([p_1, q_2])= g( [q_1, q_1+q_2-p_1]) +1. 
\]

\begin{claim}
\label{claim:g}
$g(t_1) > g(t_2)$, $g(b_1) > g(b_2)$, and $g(r_1) < g(r_2)$.
%The slopes of $t$ and $b$ are strictly smaller than the slope of  $[p_1, q_2]$ and that of $r=[r_1, r_2]$ (and $l=[l_1, l_2]$) is strictly greater than it.
\end{claim}

\begin{proof}
Since $b$ and $t$ must respectively separate $p_1$ and $q_1+q_2-p_1$ from the other two vertices of the parallelogram $\conv(q_1, p_1, q_2, q_1+q_2-p_1)$, they must respectively intersect its (parallel) edges $[p_1, q_2]$ and $[q_1, q_1+q_2-p_1]$, which implies the stated inequalities for $b$ and $t$.
The same argument  applied to the parallelogram  $\conv(p_1, q_2, p_2, p_1+p_2-p_2)$, yields the inequalities for $\ell$ and $r$.
\end{proof}

\begin{figure}[htb]
\scalebox{.75}{\input{claim4.pdf_t}}
\caption{Illustration of the proof of \Cref{claim:g}}
\label{fig:claim4}
\end{figure}


We are now ready to show a contradiction. Since the normal fan of $Q$ refines that of $P$, $Q$ must have an edge $r'$ which is a translated copy of $r$. Let $r_1'$ and $r_2'$ be its endpoints. Now consider the lattice line $d$ through $r_1'$ parallel to  $[p_1, q_2]$, that is, $g$ is constant on $d$. Let $d'$ be the parallel line defined by $g(d')=g(d)+1$. 


Consider the segment $s$ contained in $r_1' + \vecline p$ with endpoints $s_1=r_1'$ on $d$ and $s_2$ on $d'$.  
Since $t$ separates $q_1$ and $q_1+q_2-p_1$ and $g$ decreases from $t_1$ to $t_2$ (by \Cref{claim:g}), the inequality $g(x)< g(d')$ holds on $Q\cap V_2$,
and in particular for $r_2'$. Since $r_2'$ is a lattice point, $g(r_2')\leq g(d)= g(r_1')$, which contradicts \Cref{claim:g}).
\end{proof}
\giulia{add a picture?}


\begin{thebibliography}{99} 

\bibitem{BHHHJKM2019}
Matthias Beck, Christian Haase, Akihiro Higashitani, Johannes Hofscheier, Katharina Jochemko, Lukas Katth\"an, Mateusz Micha{\l}ek,
Smooth Centrally Symmetric Polytopes in Dimension 3 are IDP,
\emph{Ann.~Combin.}, in press, 2019.
\url{https://doi.org/10.1007/s00026-019-00418-x}

\bibitem{BGbook}
Winfried Bruns and Joseph Gubeladze, \emph{Polytopes, rings, and k-theory}, Monographs in Mathematics,
  Springer-Verlag, 2009, XIV, 461 p. 52 illus.
  
\bibitem{CLS}
David A.~Cox, John B.~Little, and Hal K.~Schenck.
{\em {T}oric {V}arieties}.
AMS, Providence, 2011.

 \bibitem{DLRS2010}
J. A. De Loera, J. Rambau, F. Santos
\emph{Triangulations: Structures for Algorithms and Applications}, 539 pp.
Algorithms and Computation in Mathematics, Vol. 25, Springer-Verlag. 
ISBN: 978-3-642-12970-4

\bibitem{Fakhruddin}
Najmuddin Fakhruddin.
\newblock Multiplication maps of linear systems on smooth projective toric surfaces.
  Preprint, arXiv:0208.5178, August 2002.

\bibitem{Grunbaum}
{Branko Gr\"unbaum}.
{\em {C}onvex {P}olytopes}.
Wiley, London, 1967.

\bibitem{HaaseHof}
Christian Haase and Jan Hofmann, 
Convex-normal (Pairs of) Polytopes, 
\emph{Canad. Math. Bull.} 60 (2017), 510--521.

  \bibitem{HPPS-survey}
Christian Haase, Andreas Paffenholz, Lindsay C. Piechnik, Francisco Santos. Existence of unimodular triangulations - positive results. 
%Preprint May 2014, updated December 2017, 89 pages. 
 \emph{Mem. Amer. Math. Soc.}, to appear.
 Available as arXiv preprint \href{https://arxiv.org/abs/1405.1687}{arXiv:1405.16878}
 
\bibitem{HNPS2008}
Christian Haase, Benjamin Nill, Andreas Paffenholz, and Francisco Santos, Lattice points in Minkowski sums, 
\emph{Electron.~J.~Combin.}, 15 (2008), no. 1, Note 11, 5 pp.

\bibitem{KantorSarkaria}
Jean-Michel Kantor and Karanbir~S.\ Sarkaria.
On primitive subdivisions of an elementary tetrahedron, 
\emph{Pacific J.~Math.} \textbf{211} (2003), no.~1,  123--155.

\bibitem{mfo2004}
Mini-workshop: Ehrhart Quasipolynomials: Algebra, Combinatorics, and Geometry, Oberwolfach Rep. 1 (2004), no. 3, 2071-2101, Abstracts from the mini-workshop held August 15-21, 2004, Organized by Jes\'us A.~De Loera and Christian Haase, Oberwolfach Reports. Vol. 1, no. 3. MR MR2144157

\bibitem{mfo2007}
Mini-workshop: Projective normality of smooth toric varieties, Oberwolfach Rep. 4 (2007), no. 39/2007, Abstracts from the mini-workshop held August 12-18, 2007. Organized by Christian Haase, Takayuki Hibi and Diane MacLagan.

\bibitem{Oda1997}
Tadao Oda. Problems on Minkowski sums of convex lattice polytopes. 
Abstract submitted at the Oberwolfach Conference ''Combinatorial Convexity and Algebraic Geometry'' 26.10--01.11, 1997.
Available as arXiv preprint \href{https://arxiv.org/abs/0812.1418}{arXiv:0812.1418}, 2008.

\bibitem{Ogata}
Shoetsu Ogata.
\newblock {Multiplication maps of complete linear systems on projective toric surfaces.}
\newblock {\em Interdiscip. Inf. Sci.}, 12(2):93--107, 2006.

\bibitem{Santos-hirsch}
Francisco Santos.
A counter-example to the Hirsch Conjecture.
\emph{Ann.~Math.~(2)}, 176 (July 2012), 383--412. 
DOI: 10.4007/annals.2012.176.1.7

\bibitem{SantosZiegler}
Francisco Santos and G{\"u}nter~M. Ziegler.
\newblock {Unimodular triangulations of dilated 3-polytopes}, 
\newblock {\em Trans. Moscow Math. Soc.} (2013), 293--311. % \MR{3235802}

\bibitem{Tsuchiya}
Akiyoshi Tsuchiya.
Cayley sums and Minkowski sums of 2-convex-normal lattice polytopes.
Preprint, arXiv:1804.10538, April 2018.

\bibitem{White1964}
G.~K.~White.
Lattice tetrahedra.
\emph{Canadian J.~Math.} 16 (1964), 389--396.

\end{thebibliography}








%
%\appendix
%
%\section{Old stuff}
%
%
%\begin{proof}
%%Suppose by contradiction that there is no lattice point as described in the lemma. Then there cannot be a vertex of $P$ in the corresponding strip, since $P$ is a lattice polygon. The same holds for $Q$ and its corresponding strip. Thus $P$ has two edges which each have one vertex on each side of the strip $(p_1p_2 + \langle \overrightarrow{q_1q_2}\rangle)$; we will call these edges $e_1$ and $e_2$. The same holds for $Q$ and the strip $(q_1q_2 + \langle \overrightarrow{p_1p_2}\rangle)$, and we call the edges be $f_1, f_2$. 
%
%Let us fix coordinates in which $p_1p_2$ and $q_1q_2$ are a vertical and horizontal segment of length one intersecting at the origin. That is, there are numbers $\alpha, \beta \in (0,1)$ such that
%\[
%p_1=(0,-\alpha), \quad
%p_2=(0,1-\alpha), \quad
%q_1=(-\beta,0), \quad
%q_2=(1-\beta,0).
%\] 
%The ambient lattice is a certain lattice containing $p_1,p_2,q_1,q_2$ (the case of index two is when $\alpha=\beta=1/2$ and the lattice is spanned by these four points).
%
%Suppose by contradiction that there is no lattice point as described in the lemma. Then there cannot be a vertex of $P$ in the corresponding strip, since $P$ is a lattice polygon. The same holds for $Q$ and its corresponding strip. Thus $P$ has two primitive edges which each have one vertex on each side of the strip $p_1p_2 + \langle \overrightarrow{q_1q_2}\rangle$; we will call these edges $\ell$ and $r$ (with $\ell$ to the left and $r$ to the right). The same holds for $Q$ and the strip $q_1q_2 + \langle \overrightarrow{p_1p_2}\rangle$, and we call the edges  $b, t$ (with $b$ below and $t$ above). To make things more explicit, we assume the following coordinates for the edge-vectors:
%\[
%\vec l = (l_1,l_2), \quad
%\vec r= (r_1,r_2), \quad
%\vec b= (b_1,b_2), \quad
%\vec t= (t_1,t_2).
%\]
%Without loss of generality we assume $l_2,r_2,b_1,t_1 >0$. (They are non-zero and we can arbitrarily choose a sign for each vector).
%
%
%
%
%
%
%Since the normal fan of $Q$ refines that of $P$, $Q$ will contain edges $e_1'$ and $e_2'$ parallel to $e_1$ and $e_2$ respectively. These edges of $Q$ are all distinct and appear alternating in $Q$, that is, they appear in the order $f_1, e_1', f_2, e_2'$. Indeed, consider the quadrilateral $S$ with vertices $q_1, p_1+l, q_2, p_2+l$, where $l$ is a lattice point chosen so that the segments $q_1q_2$ and $p_1+l,p_2+l$ intersect in their interior; we can further require that $\conv( q_1, p_1+l, q_2)$ is unimodular \giulia{not sure if we use it}. $S$ lies inside the $Q$-strip, while its translate $S-l$ lies inside the $P$-strip. Thus in order to have no lattice points in the strips, we must have that the edge $f_i$ separates the vertex $p_i +l$ of $S$ from the remaining three, and  the same for the edge $e_i +l$ and vertex $q_i$ of $S$. This is equivalent to saying that the normals to the edges $f_1, e_1, f_2, e_2$ lie in different open cones of the normal fan of $S$, in this cyclic order. 
%
%- both $f_1$ or $f_2$ must meet the line through $q_1$ and $q_2$, else they would not cut out the lattice point they need to. same goes for $e_1$ and $e_2$ and the line through $p_1$ and $p_2$.
%
%claim: the segments $e_1 + l$ and $e_2 + l$ respectively intersect both $f_1$ and $f_2$. 
%
%claim: if we tilt one of the $f_i$s so that they are parallel, there is no room to fit $e_i$, since it intersects both. 
%
%
%
%
%
%%Consider the segments on the lattice lines parallel to $p_1 p_2$ cut out by the edges $f_1$ and $f_2$ \giulia{this should be formulated more comprehensibly}. Since these segments do not contain lattice points, they must have lattice length stricly less than $1$ (or the width wrt $q$ is less than $k$, the index of $p,q$), which means that the smallest width in direction of $p_1 p_2$ between a "consecutive" vertices of $f_1$ and of $f_2$ must also be strictly less than $k$. 
%
%%Further, we observe that both vertices of $f_1$ (or $f_2$) cannot lie strictly above the line through $q_1$ and $q_2$, else the strip would contain a lattice point. 
%\end{proof}
%
%
%
%\begin{figure}[htb]
%  \centering
%\begin{tikzpicture}[mynode/.style={circle, inner sep=1.5pt, fill=red}, scale =
%  .7]
%%coordinate axes
%   \tkzInit[xmax=3.5,ymax=3.5,xmin=-3.5,ymin=-3.5]
%%        \tkzLabelX[orig=false,label options={font=\tiny}]
%%        \tkzLabelY[orig=false,label options={font=\tiny}]
%        \tkzGrid
%%       \tkzDrawX
%%       \tkzDrawY
%
%%hexagon
%  \node[mynode] (A) at (0,-1) {}; 
%%  \node[mynode] (B) at (1,1) {};  
%  \node[mynode] (C) at (2,3) {};  
%  \node[mynode] (D) at (0,1) {};  
%%  \node[mynode] (E) at (-1,-1) {};  
%  \node[mynode] (F) at (-2,-3) {};  
%
%%other points
%%  \node[mynode] () at (0,-1) {}; 
%
%
%  \draw[black] (A) -- (C) -- (D) -- (F) -- (A);
%  \draw[black, thick] (A) -- (D);
%\end{tikzpicture}
%%
%\begin{tikzpicture}[mynode/.style={circle, inner sep=1.5pt, fill=red}, scale =
%  .7]
%%coordinate axes
%   \tkzInit[xmax=3.5,ymax=3.5,xmin=-3.5,ymin=-3.5]
%%        \tkzLabelX[orig=false,label options={font=\tiny}]
%%        \tkzLabelY[orig=false,label options={font=\tiny}]
%        \tkzGrid
%%        \tkzDrawX
%%        \tkzDrawY
%
%%hexagon
%  \node[mynode] (A) at (1,0) {}; 
%  \node[mynode] (B) at (2,1) {};  
%  \node[mynode] (C) at (3,3) {};  
%  \node[mynode] (D) at (-1,0) {};  
%  \node[mynode] (E) at (-2,-1) {};  
%  \node[mynode] (F) at (-3,-3) {};  
%
%  \node[mynode] (p1) at (-1,0) {}; 
%  \node[mynode] (p2) at (1,0) {};  
%
%  \draw[black] (A) -- (B) -- (C) -- (D) -- (E) -- (F) -- (A);
%  \draw[black, thick] (p1) -- (p2);
%\end{tikzpicture}
%
%\centerline{\hfill $P$ \hfill\hfill $Q$ \hfill}
%
%\caption{Partial counter-example to \Cref{lemma:cayley}. With respect to the standard lattice it is not a counter-example because (a) the edges $p_1p_2$ and $q_1q_2$ are not primitive and (b) the origin is a lattice point in the interior of both bands (but it is the only one). 
%%
%With respect to the lattice of index two it is not a counter-example either, because two vertices of $Q$ are not in that lattice.
%}
%\label{fig:width3}
%\end{figure}


\end{document}

